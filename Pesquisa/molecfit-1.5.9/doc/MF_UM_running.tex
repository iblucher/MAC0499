%-------------------------------------------------------------------------------
\section{Running procedure}\label{sec:running}
%-------------------------------------------------------------------------------
%-------------------------------------------------------------------------------
\subsection{Environment variables {\tt molecfit}}\label{sec:variables}
%-------------------------------------------------------------------------------
Before the execution you can to define some environment variables in order to 
change the behavior of molecfit.
\begin{itemize}
  \item TMPDIR: Define the folder where create the temporary directory.
        If it is not define, molecfit use the default 'tmp' directory in the system
  \item MOLECFITDIR: Define the installation directory.
        If it is not define, molecfit use the path of the installation directory provide in compilation time.
  \item MOLECFITDIR\_DATA: Define the data directory.
        If it is not define, molecfit use the path of the data directory create in the installation.
\end{itemize}
Example of use:
\begin{verbatim}
    export TMPDIR=$PWD
    export MOLECFITDIR=$PWD/dest
    export MOLECFITDIR_DATA=$MOLECFITDIR/share/molecfit/data
\end{verbatim}

%-------------------------------------------------------------------------------
\subsection{Calling {\tt molecfit}}\label{sec:calling}
%-------------------------------------------------------------------------------
After the compilation, the executable {\tt molecfit} for the fitting of
molecular transmission and emission is located in the {\tt bin/} directory. It
can be invoked via
% \begin{verbatim}
%     cd <basedir>
%     ./bin/molecfit <parameter file> <optional mode>
% \end{verbatim}
% or
\begin{verbatim}
    cd <arbitrary_dir>
    <fullpath>/bin/molecfit <parameter file> <optional mode>
\end{verbatim}
where {\tt <parameter file>} represents a user-defined parameter file
(including paths). The {\tt <optional mode>} parameter enables the user to run
different modes of the fitting procedure:
\begin{itemize}
    \item `m' - multiple: Choosing this option, the fitting procedure follows
          all five steps described in Section~\ref{sec:algorithm}.
    \item `s' - single run: Only step \#5 is carried out (see
          Section~\ref{sec:algorithm}).
\end{itemize}
Note that in both modes fitting steps that were designated to be skipped (by
setting fit flag $= 0$ in the parameter file) are excluded (see
Section~\ref{sec:paramfile}).

The {\tt <INST\_DIR>/examples} directory contains several examples,
including input spectra (in subfolder {\tt input}) and configuration parameter
files (in subfolder {\tt config}) for the CRIRES, VISIR and X-Shooter
instruments. The CRIRES spectrum represents the best data set for analysing
molecular abundances and is optimally fitted using the parameter file listed
in Section~\ref{sec:paramfile}:
% \begin{verbatim}
%     cd <basedir>
%     ./bin/molecfit examples/config/molecfit_crires.par
% \end{verbatim}
\begin{verbatim}
    cd <arbitrary_dir>
    <fullpath>/bin/molecfit examples/config/molecfit_crires.par
\end{verbatim}
The other examples can be started synonymously.

The conversion of the input data file into a FITS table that is suitable
for the fitting procedure can also be performed by means of the executable
{\tt preptable} plus {\tt <parameter file>} as input parameter. The procedure
is the same as for {\tt molecfit}, but it stops before the fitting is started.

%-------------------------------------------------------------------------------
\subsection{Calling {\tt calctrans}}\label{sec:callingct}
%-------------------------------------------------------------------------------
After the compilation, the executable {\tt calctrans} for the derivation of
the telluric absorption spectrum over the whole spectrum is also located in the {\tt bin/}
directory. It can be run by
% \begin{verbatim}
%     cd <basedir>
%     ./bin/calctrans <parameter file>
% \end{verbatim}
% or
\begin{verbatim}
    cd <arbitrary_dir>
    <fullpath>/bin/calctrans <parameter file>
\end{verbatim}
where {\tt <parameter file>} represents the same user-defined parameter file
(including paths) as used for {\tt molecfit}.

The  telluric  absorption  correction  can be  calculated  for  a  new
spectrum with a different aimass (cf.  Section~\ref{sec:callingct}) by
changing the argument of {\sc  filename} in the {\tt <parameter file>}
(see Section~\ref{sec:params}). If it is an ASCII file, the argument 
of {\sc telalt} must be also changed.

Note  that the  line  kernel  and the  wavelength  correction are  not
modified  by this  approach.  This  requires a  direct  fit with  {\tt
  molecfit}.


%-------------------------------------------------------------------------------
\subsection{Calling {\tt corrfilelist}}\label{sec:callingcfl}
%-------------------------------------------------------------------------------
The {\tt bin/} directory contains the executable {\tt corrfilelist}
for the telluric absorption correction for each of the spectra listed in
an ASCII file given as argument to the {\sc listname} keyword 
in the parameter file. Since the
correction function is not recalculated, airmass differences are not
considered (cf. Section~\ref{sec:callingct}). In the same way as
{\tt calctrans}, {\tt corrfilelist} can be invoked by
% \begin{verbatim}
%     cd <basedir>
%     ./bin/corrfilelist <parameter file>
% \end{verbatim}
% or
\begin{verbatim}
    cd <arbitrary_dir>
    <fullpath>/bin/corrfilelist <parameter file>
\end{verbatim}
where {\tt <parameter file>} represents the same user-defined parameter file
(including paths) as used for {\tt molecfit}.

%-------------------------------------------------------------------------------
\subsection{Calling {\tt calctrans\_lblrtm} and {\tt calctrans\_convolution}}\label{sec:callingct2}
%-------------------------------------------------------------------------------

Alternatively, correction of spectra affected by different (but known)
kernels or different airmasses can  be dealt with in  the following way by  calling two other
executables:
\begin{verbatim}
   cd <arbitrary_dir>
\end{verbatim}
\begin{itemize}
\item 
\begin{verbatim}
<fullpath>/bin/calctrans_lblrtm  <parameter file>
\end{verbatim}
  calculates  the transmission  spectrum from LBLRTM
  using the best fit parameter over the whole wavelength range, with a
  sampling 5 times better than the input spectrum;
\item 
\begin{verbatim}
<fullpath>/bin/calctrans_convolution  <parameter file>
\end{verbatim}
performs the  convolution of  the
  spectrum produced by  {\tt calctrans\_lblrtm} with  the user-provided kernel.
\end{itemize}

A typical use is the following:
\begin{enumerate}
\item the user knows well the  kernel for the spectra provided by each
  fiber of  a multi-fiber spectrograph.   {\tt Molecfit} is  called to
  fit the  molecular content  on an input  spectrum (argument  of {\sc
    filename})  corresponding to  a  given fiber;  the parameter  file
  includes  the filename  with the  corresponding kernel  (argument of
  {\sc kernel\_file});
\item once a satisfactory fit is obtained, {\tt calctrans\_lblrtm} is
  run with the parameter file unchanged;
\item then {\tt calctrans\_convolution} is called with the same
  parameter file in which the name of the input file (argument of
  {\sc filename}) has been changed for the spectrum of another fiber, and
  where the kernel filename (argument of {\sc kernel\_file}) has been changed
  to provide the corresponding kernel. This step can be repeated for each
  spectra.
\end{enumerate}
On  the  other  hand,  if   the  airmass  for  another  fiber  differs
significantly from the first, {\tt calctrans\_lblrtm} should be called
with  the  same parameter  file  where  the  name  of the  input  file
(argument of {\sc  filename}) has been changed to the  filename of the
corresponding  spectrum.  The  telescope altitude  (argument  of  {\sc
  telalt}) must be changed also if the input file is an ASCII file.



%-------------------------------------------------------------------------------
\subsection{The parameter file}\label{sec:paramfile}
%-------------------------------------------------------------------------------
All parameters needed for the fit are read from an ASCII parameter file, which
contains parameter names, descriptions, and initial values. Example files can
be found in the folder {\tt examples/config/}.

The file is divided into sections, where parameters belonging together are
grouped. These sections incorporate the directory structure (labelled
{\tt DIRECTORY STRUCTURE}), the required input ({\tt INPUT DATA}), output
options ({\tt RESULTS}), the precision of the fit ({\tt FIT PRECISION}),
molecular information ({\tt MOLECULAR COLUMNS}), the fit of the background and
the continuum ({\tt BACKGROUND AND CONTINUUM}), the wavelength fit
({\tt WAVE\-LENGTH SOLUTION}), the resolution fit ({\tt RESOLUTION}),
environmental parameters ({\tt AMBIENT}\linebreak {\tt PARAMETERS}), and the
atmospheric profiles ({\tt ATMO\-SPHERIC PROFILES}). The individual parameters
are given one per line. The parameter name plus a trailing colon and space is
followed by one or more parameter values, which also have to be separated by
spaces. Lines beginning with a hash key are assumed to be comments and are
skipped. Thus, next to removing parameter lines, there are several options to
``deactivate'' a parameter in the parameter file:

Command ``active'':
\begin{verbatim}
ftol: 1e-2
\end{verbatim}

Command ``deactivated'' (commented out):
\begin{verbatim}
#ftol: 1e-2
\end{verbatim}

Command ``deactivated'' (no trailing colon):
\begin{verbatim}
ftol 1e-2
\end{verbatim}

Command ``deactivated'' (no parameter value)
\begin{verbatim}
ftol:
\end{verbatim}
or
\begin{verbatim}
ftol
\end{verbatim}

In the following, {\tt examples/config/molecfit\_crires.par} is shown as
an example of a parameter file. Individual parameters are explained in
Section~\ref{sec:params}:

{\small
\begin{verbatim}
### Driver for MOLECFIT

## INPUT DATA

# Data file name (path relative to the current directory or absolute path)
filename: examples/input/crires_spec_jitter_extracted_0000.fits

# ASCII list of files to be corrected for telluric absorption using the
# transmission curve derived from the input reference file (path of list and
# listed files relative to the current directory or absolute path; default: "none")
listname: none

# Type of input spectrum -- 1 = transmission (default); 0 = emission
trans: 1

# Names of the file columns (table) or extensions (image) containing:
# Wavelength  Flux  Flux_Err  Mask
# - Flux_Err and/or Mask can be avoided by writing 'NULL'
# - 'NULL' is required for Wavelength if it is given by header keywords
# - parameter list: col_lam, col_flux, col_dflux, and col_mask
columns: Wavelength Extracted_OPT Error_OPT NULL

# Default error relative to mean for the case that the error column is missing
default_error: 0.01

# Multiplicative factor to convert wavelength to micron
# (e.g. nm -> wlgtomicron = 1e-3)
wlgtomicron: 1e-3

# Wavelengths in vacuum (= vac) or air (= air)
vac_air: vac

# ASCII or FITS table for wavelength ranges in micron to be fitted
# (path relative to the current directory or absolute path; default: "none")
wrange_include: none

# ASCII or FITS table for wavelength ranges in micron to be excluded from the
# fit (path relative to the current directory or absolute path; default: "none")
wrange_exclude: none

# ASCII or FITS table for pixel ranges to be excluded from the fit
# (path relative to the current directory or absolute path; default: "none")
prange_exclude: examples/config/exclude_crires.dat

## RESULTS

# Directory for output files (path relative to the current directory or absolute path)
output_dir: output

# Name for output files
# (supplemented by "_fit" or "_tac" as well as ".asc", ".atm", ".fits",
# ".par, ".ps", and ".res")
output_name: molecfit_crires

# Plot creation: gnuplot is used to create control plots
# W - screen output only (incorporating wxt terminal in gnuplot)
# X - screen output only (incorporating x11 terminal in gnuplot)
# P - postscript file labelled '<output_name>.ps', stored in <output_dir>
# combinations possible, i.e. WP, WX, XP, WXP (however, keep the order!)
# all other input: no plot creation is performed
plot_creation: XP

# Create plots for individual fit ranges? -- 1 = yes; 0 = no
plot_range: 0

## FIT PRECISION

# Relative chi2 convergence criterion
ftol: 1e-2

# Relative parameter convergence criterion
xtol: 1e-2

## MOLECULAR COLUMNS

# List of molecules to be included in the model
# (default: 'H2O', N_val: nmolec)
list_molec: H2O CH4 O3

# Fit flags for molecules -- 1 = yes; 0 = no (N_val: nmolec)
fit_molec: 1 1 1

# Values of molecular columns, expressed relatively to the input ATM profile
# columns (N_val: nmolec)
relcol: 1. 1. 1.

## BACKGROUND AND CONTINUUM

# Conversion of fluxes from phot/(s*m2*mum*as2) (emission spectrum only) to
# flux unit of observed spectrum:
# 0: phot/(s*m^2*mum*as^2) [no conversion]
# 1: W/(m^2*mum*as^2)
# 2: erg/(s*cm^2*A*as^2)
# 3: mJy/as^2
# For other units, the conversion factor has to be considered as constant term
# of the continuum fit.
flux_unit: 0

# Fit of telescope background -- 1 = yes; 0 = no (emission spectrum only)
fit_back: 0

# Initial value for telescope background fit (range: [0,1])
telback: 0.1

# Polynomial fit of continuum --> degree: cont_n
fit_cont: 1

# Degree of coefficients for continuum fit
cont_n: 3

# Initial constant term for continuum fit (valid for all fit ranges)
# (emission spectrum: about 1 for correct flux_unit)
cont_const: 1.

## WAVELENGTH SOLUTION

# Refinement of wavelength solution using a polynomial of degree wlc_n
fit_wlc: 1

# Polynomial degree of the refined wavelength solution
wlc_n: 3

# Initial constant term for wavelength correction (shift relative to half
# wavelength range)
wlc_const: 0.

## RESOLUTION

# Fit resolution by boxcar -- 1 = yes; 0 = no
fit_res_box: 0

# Initial value for FWHM of boxcar relative to slit width (>= 0. and <= 2.)
relres_box: 0.

# Voigt profile approximation instead of independent Gaussian and Lorentzian
# kernels? -- 1 = yes; 0 = no
kernmode: 0

# Fit resolution by Gaussian -- 1 = yes; 0 = no
fit_res_gauss: 1

# Initial value for FWHM of Gaussian in pixels
res_gauss: 1.

# Fit resolution by Lorentzian -- 1 = yes; 0 = no
fit_res_lorentz: 0

# Initial value for FWHM of Lorentzian in pixels
res_lorentz: 0.5

# Size of Gaussian/Lorentzian/Voigtian kernel in FWHM
kernfac: 300.

# Variable kernel (linear increase with wavelength)? -- 1 = yes; 0 = no
varkern: 0

# ASCII file for kernel elements (one per line; normalisation not required)
# instead of synthetic kernel consisting of boxcar, Gaussian, and Lorentzian
# components (path relative to the current directory or absolute path; default: "none")
kernel_file: none

## AMBIENT PARAMETERS

# If the input data file contains a suitable FITS header, the keyword names of
# the following parameters will be read, but the corresponding values will not
# be used. The reading of parameter values from this file can be forced by
# setting keywords to NONE.

# Observing date in years or MJD in days
obsdate
obsdate_key: MJD-OBS

# UTC in s
utc
utc_key: UTC

# Telescope altitude angle in deg
telalt
telalt_key: ESO TEL ALT

# Humidity in %
rhum
rhum_key: ESO TEL AMBI RHUM

# Pressure in hPa
pres
pres_key: ESO TEL AMBI PRES START

# Ambient temperature in deg C
temp
temp_key: ESO TEL AMBI TEMP

# Mirror temperature in deg C
m1temp
m1temp_key: ESO TEL TH M1 TEMP

# Elevation above sea level in m (default is Paranal: 2635m)
geoelev
geoelev_key: ESO TEL GEOELEV

# Longitude (default is Paranal: -70.4051)
longitude
longitude_key: ESO TEL GEOLON

# Latitude (default is Paranal: -24.6276)
latitude
latitude_key: ESO TEL GEOLAT

## INSTRUMENTAL PARAMETERS

# Slit width in arcsec (taken from FITS header if present)
slitw: 0.4
slitw_key: ESO INS SLIT1 WID

# Pixel scale in arcsec (taken from this file only)
pixsc: 0.086
pixsc_key: NONE

## ATMOSPHERIC PROFILES

# Reference atmospheric profile
ref_atm: equ.atm

# Specific GDAS-like input profile (P[hPa] HGT[m] T[K] RELHUM[%]) (path
# relative to the installation directory or absolute path). In the case of "none", no GDAS
# profiles will be considered. The default "auto" performs an automatic
# retrieval.
gdas_prof: auto

# Grid of layer heights for merging ref_atm and GDAS profile. Fixed grid = 1
# (default) and natural grid = 0.
layers: 1

# Upper mixing height in km (default: 5) for considering data of a local meteo
# station. If emix is below geoelev, rhum, pres, and temp are not used for
# modifying the corresponding profiles.
emix: 5

# PWV value in mm for the input water vapour profile. The merged profile
# composed of ref_atm, GDAS, and local meteo data will be scaled to this value
# if pwv > 0 (default: -1 -> no scaling).
pwv: -1

end
\end{verbatim}
}

%\pagebreak

%-------------------------------------------------------------------------------
\subsection{The parameters}\label{sec:params}
%-------------------------------------------------------------------------------
In the following, the individual parameters are explained in more detail by
following the order as they appear in the parameter file.

% {\bf\large\tt\#\# DIRECTORY STRUCTURE:}
% \begin{itemize}
% \item {\sc basedir}: Base directory (default: ".") for the following folder
% structure:
% \begin{verbatim}
%                |--bin/
%                |
%                |--config/
%   <basedir>----|
%                |--data/
%                |
%                |--output/
% \end{verbatim}
% A relative or an absolute path can be provided. In the former case, \mf\
% has to be started in <basedir>. The subfolder ``bin/'' of <basedir> is to
% contain all required executables (see Section~\ref{sec:installation}).
% Configuration files for the radiative transfer codes and mandatory input data
% (line parameter database, atmospheric profiles, or molecular cross section
% files) have to be located in ``config/'' and ``data/'', respectively. Finally,
% ``output/'' is the default folder for all resulting files described in
% Section~\ref{sec:output}. The output directory can be modified by the parameter
% {\sc output\_dir} (see below). For example \mf\ parameter files and input
% spectra, we have created the directory ``examples/'' with the subfolders
% ``config/'' and ``input/''. The names of these folders are arbitrary and their
% presence is not expected by the code.
% \end{itemize}

{\bf\large\tt\#\# INPUT DATA:}
\begin{itemize}
\item {\sc filename}: Path and name of the input spectrum. The path has to be
relative to the current directory or absolute. The accepted file formats for the
input spectrum are ASCII and FITS. The spectral data of the latter has to be
provided either in table form (multiple FITS extensions for different chips) or
as 1D image (with optional error and quality data in additional FITS
extensions).  See section \ref{sec:inputspec} for more information about the
input file format.
\item {\sc listname}: Input ASCII list of files for {\tt corrfilelist}, which
corrects spectra for telluric absorption using the transmission curve derived
by {\tt calctrans}. The path to the list and the listed files has to be
relative to the current directory or absolute. In contrast to {\sc filename}, the
listed files can also be 2D FITS images. By default, no file list is expected
(``none'').
\item {\sc trans}: type of input spectrum (1 = transmission, 0 = emission;
default = 1).
\item {\sc columns}: Column names of the input file containing information on
wavelength, flux, flux\_err, and mask. The latter two are optional and can be
disabled by setting them to ``NULL''. For ASCII files, the column names are
irrelevant, with the exception of ``NULL'' input. In the case of FITS images,
the given labels are compared to the FITS extension names (keyword
``EXTNAME''). Since it is expected that the flux data is in the first layer of
the FITS file (0th extension), the existence of a corresponding FITS keyword is
not required for this column. Internally, the code denotes the columns as
{\sc col\_lam}, {\sc col\_flux}, {\sc col\_dflux}, and {\sc col\_mask}.
\item {\sc default\_error}: Default error relative to the mean in case
the error column is missing (column name = ``NULL'', see previous
record).
\item {\sc wlgtomicron}: Multiplicative factor to convert input wavelength unit
into $\mu$m. For example, for nm this parameter has to set to $10^{-3}$.
\item {\sc vac\_air}: Wavelengths in vacuum (= ``vac'') or air (= ``air'')?
This parameter depends on the instrument and the wavelength calibration
approach.
\item {\sc wrange\_include}: ASCII or FITS table for wavelength ranges to be
fitted. The path to the file has to be relative to the current directory or absolute.
Except for empty lines and comment lines starting with \#, each line of an
ASCII file has to provide a lower and an upper wavelength limit in $\mu$m. FITS
tables must contain exactly two columns. The column labels are not fixed. If a
range table is not desired and the full spectrum shall be fitted, ``none'' can
be given. This is also the default value.
\item {\sc wrange\_exclude}: ASCII or FITS table for wavelength ranges to be
excluded from the fit. The path to the file has to be relative to the current directory
or absolute. Except for empty lines and comment lines starting with \#, each
line of an ASCII file has to provide a lower and an upper wavelength limit in
$\mu$m. FITS tables must contain exactly two columns. The column labels are not
fixed. If the exclusion of wavelength ranges is not desired, ``none'' can be
given. This is also the default value.
\item {\sc prange\_exclude}: ASCII or FITS table for pixel ranges to be
excluded from the fit. The path to the file has to be relative to the current directory
or absolute. Except for empty lines and comment lines starting with \#, each
line of an ASCII file has to provide a lower and an upper pixel. FITS tables
must contain exactly two columns. The column labels are not fixed. The pixel
counting starts with 1. If there are several chips, the pixel counting
continues at chip limits. If a range table is not desired and all pixels shall
be fitted, ``none'' can be given. This is also the default value.
\end{itemize}

{\bf\large\tt\#\# RESULTS:}
\begin{itemize}
\item {\sc output\_dir}: Output directory as path relative to the current directory or
absolute path (see also Section {\tt DIRECTORY STRUCTURE)}. The folder is
created if it does not exist.
\item {\sc output\_name}: Unique name space for output files. The extensions
are supplemented by ``\_fit'' or ``\_tac'' as well as ``.asc'', ``.atm'',
``.fits'', ``.par'', ``.ps'', and ``.res''. See Section~\ref{sec:output} for
more details.
\item {\sc plot\_creation}: {\tt molecfit} invokes {\tt gnuplot} to create
control plots for verifying the fit. For each fit range, an individual plot
is created, which contains a comparison of the observed and the modelled
spectrum and their residual. An overview plot for the entire wavelength range
is also produced. Finally, {\tt calctrans} creates a plot that shows the
initial and the telluric absorption corrected spectrum. This parameter allows
specifying the output terminal of gnuplot:
W - screen output incorporating wxt terminal,
X - screen output incorporating x11 terminal,
P - postscript files labelled `<output\_name>\_fit.ps',
    `<output\_name>\_fit\_<fitrange>.ps', and `<output\_name>\_tac.ps' (stored
    in <output\_dir>).
Also, arbitrary combinations are possible, \eg\ WP for wxt terminal and
postscript, or WX for wxt and x11 terminal output.
\item {\sc plot\_range}: Flag for creation of plots of individual fit ranges
(1 = yes, 0 = no). If the flag is set to 0, the files labelled
`<output\_name>\_fit\_<fitrange>.ps' are not produced.
\end{itemize}

{\bf\large\tt\#\# FIT PRECISION:}
\begin{itemize}
\item {\sc ftol}: relative $\chi^2$ convergence criterion.
\item {\sc xtol}: relative parameter convergence criterion.
\end{itemize}

{\bf\large\tt\#\# MOLECULAR COLUMNS:}
\begin{itemize}
\item {\sc list\_molec}: List of molecules separated by blanks, which are
included in the model. Note that only those molecules are valid, which are
also present in the standard atmospheric profile {\sc ref\_atm} (see also
Section~\ref{sec:mipas}).
\item {\sc fit\_molec}: Fit flags (1 = yes, 0 = no) separated by blanks for
list of molecules. \# has to match number of molecules in {\sc list\_molec}.
Example: {\sc 1 0 1} implies that only the first and third molecule are
fitted. The middle molecule is included statically in the fit, \ie\ its
abundance is not varied in the fit.
\item {\sc relcol}: Relative molecular column densities separated by blanks,
normalised to the values in the input atmospheric profile (1 = 100\%). \# has
to match number of molecules in {\sc list\_molec}.
\end{itemize}

{\bf\large\tt\#\# BACKGROUND AND CONTINUUM:}
\begin{itemize}
\item {\sc flux\_unit}: Conversion of fluxes from phot/(s$\cdot$m$^2\cdot$
$\mu$m$\cdot$arcsec$^2$) (emission spectrum only) to flux unit of observed
spectrum: 0 = phot/(s$\cdot$m$^2\cdot$$\mu$m$\cdot$arcsec$^2$) (--> no
conversion), 1 = W/(m$^2\cdot$$\mu$m$\cdot$arcsec$^2$),
2 = erg/(s$\cdot$cm$^2\cdot$\AA$\cdot$arcsec$^2$), 3 = mJy/arcsec$^2$.
For other units differing from the offered units by a constant factor, the
constant term of the continuum fit parameters {\sc cont\_const} has to be
multiplied by the conversion factor. For non-flux-calibrated spectra provided
in analogue-to-digital units (ADU), the default 0 is the best choice.
\item {\sc fit\_back}: fit of telescope background (1 = yes, 0 = no); used for
emission spectrum fit only.
\item {\sc telback}: initial emissivity value for telescope background grey
body fit; used for emission spectrum fit only.
\item {\sc fit\_cont}: Flag for polynomial fit of the continuum (1 = yes,
0 = no). For each fit range/chip, the model spectrum is multiplied by the
resulting polynomial.
\item {\sc cont\_n}: degree of polynomial for the continuum fit.
\item {\sc cont\_const}: Initial constant term of the polynomial for the
continuum fit. The same value is used for all fit ranges/chips. By default,
1 is assumed. Since all higher terms are set to 0 at the beginning, the fitting
procedure starts without a continuum correction if the default setting is used.
For sky emission spectra with correct {\sc flux\_unit}, the true continuum
correction factor should be very close to the default value 1.
\end{itemize}

{\bf\large\tt\#\# WAVELENGTH SOLUTION:}
\begin{itemize}
\item {\sc fit\_wlc}: flag for refinement of wavelength solution (1 = yes,
0 = no).
\item {\sc wlc\_n}: degree of Chebyshev polynomial for refined wavelength
solution.
\item {\sc wlc\_const}: Constant term of the Chebyshev polynomial for the
wavelength solution, which is derived for each chip independently. The provided
constant term is used for all chips. The given value represents a shift
relative to half the wavelength range of the input spectrum. By default,
0 is assumed. Since the linear term and the higher terms are set to 1 and 0,
respectively, at the beginning, the fitting procedure starts without a
wavelength correction if the default setting is used.
\end{itemize}

{\bf\large\tt\#\# RESOLUTION:}\\[0.5cm]
The resolution fit incorporates a combined, boxcar, Gaussian, and
Lorentzian convolution. The latter two convolutions result in a Voigt profile.
\begin{itemize}
\item {\sc fit\_res\_box}: flag for resolution fit using a boxcar filter
(1 = yes, 0 = no).
\item {\sc relres\_box}: Initial value for FWHM of boxcar relative to slit
width ($\geq 0$ and $\leq 2$). A value of 0 combined with
{\sc fit\_res\_box}~=~0 switches off the convolution of a boxcar.
\item {\sc kernmode}: By default (= 0), spectra are convolved by independent
Gaussian and Lorentzian kernels. It is also possible to perform only one
convolution with a kernel derived from a Voigt profile approximation (= 1),
which also uses the FWHM of Gaussian and Lorentzian as input.
\item {\sc fit\_res\_gauss}: flag for resolution fit using a Gaussian filter
(1 = yes, 0 = no).
\item {\sc res\_gauss}: Initial value for FWHM of Gaussian (in pixels). A value
of 0 combined with \\ {\sc fit\_res\_gauss}~=~0 switches off the convolution of
a Gaussian.
\item {\sc fit\_res\_lorentz}: flag for resolution fit using a Lorentzian
filter (1 = yes, 0 = no).
\item {\sc res\_lorentz}: Initial value for FWHM of Lorentzian (in pixels). A
value of 0 combined with \\ {\sc fit\_res\_lorentz}~=~0 switches off the
convolution of a Lorentzian.
\item {\sc kernfac}: Kernel size in units of FWHM. Depending on
{\sc kernmode} either the Gaussian and Lorentzian kernels or the combined
Voigt profile kernel are relevant. By default, a value of 3 is assumed.
\item {\sc varkern}: Flag for selecting a constant (= 0) or a variable kernel
(= 1). In the latter case, all FWHM values are related to the central
wavelength of the full wavelength range. The variable kernel increases linearly
with wavelength, \ie\ the resolution is constant. X-Shooter Echelle spectra
show this behaviour. For spectra where the object profile in the slit mainly
determines the kernel, the default constant kernel option is recommended.
\item {\sc kernel\_file} ASCII file for fixed kernel elements (pixels) instead
of a synthetic kernel consisting of boxcar, Gaussian, and Lorentzian
components. The latter is the default case and requires ``none''. The path to
the file has to be relative to the current directory or absolute. In the file, each
kernel element has to be given on a separate line. It is not required that the
kernel has been normalised to 1. Note that a given {\sc kernel\_file} overrules
all other parameters related to the line profile. The kernel will not be
fitted and will not depend on the wavelength (irrespective of {\sc varkern}).
\end{itemize}

{\bf\large\tt\#\# AMBIENT PARAMETERS:}\\[0.5cm]
The parameters of this section are only required if the input spectrum is
provided as ASCII file or if the ESO FITS header keywords are missing or differ
from the default names. The use of a specific value from the parameter file can
be forced by setting the keyword name to ``NONE''. If no keyword is provided
by the input file (\eg\ ASCII file), it is not required to set the keyword
names to ``NONE''.
\begin{itemize}
\item {\sc obsdate}: Observing date in years, \eg\ 2008.566, or MJD in days.
This parameter is required for the retrieval of GDAS data.
\item {\sc obsdate\_key}: FITS keyword name for {\sc obsdate} (default:
``MJD-OBS'').
\item {\sc utc}: UTC in s, starting at 00:00. This parameter is required for
the retrieval of GDAS data.
\item {\sc utc\_key}: FITS keyword name for {\sc utc} (default: ``UTC'').
\item {\sc telalt}: altitude angle of telescope in deg.
\item {\sc telalt\_key}: FITS keyword name for {\sc telalt} (default:
``ESO TEL ALT'').
\item {\sc rhum}: Relative humidity in \% for {\sc geoelev}. This parameter is
only relevant if {\sc emix} is larger than {\sc geoelev}.
\item {\sc rhum\_key}: FITS keyword name for {\sc rhum} (default:
``ESO TEL AMBI RHUM'').
\item {\sc pres}: Pressure in hPa for {\sc geoelev}. This parameter is only
relevant if {\sc emix} is larger than {\sc geoelev}.
\item {\sc pres\_key}: FITS keyword name for {\sc pres} (default:
``ESO TEL AMBI PRES START'').
\item {\sc temp}: Ambient temperature in $^\circ$C for {\sc geoelev}. This
parameter is only relevant if {\sc emix} is larger than {\sc geoelev}.
\item {\sc temp\_key}: FITS keyword name for {\sc temp} (default:
``ESO TEL AMBI TEMP'').
\item {\sc m1temp}: Temperature of primary mirror M1 in $^\circ$C. This
parameter is only relevant for emission spectra (\ie\ {\sc trans}~=~0), where
the thermal emission of the telescope has to be considered.
\item {\sc m1temp\_key}: FITS keyword name for {\sc m1temp} (default:
``ESO TEL TH M1 TEMP'').
\item {\sc geoelev}: elevation above sea level in m (default is Paranal:
2635\,m).
\item {\sc geoelev\_key}: FITS keyword name for {\sc geoelev} (default:
``ESO TEL GEOELEV'').
\item {\sc longitude}: Longitude in deg (default is Paranal: -70.4051). This
parameter is required for the retrieval of GDAS data.
\item {\sc longitude\_key}: FITS keyword name for {\sc longitude} (default:
``ESO TEL GEOLON'').
\item {\sc latitude}: Latitude in deg (default is Paranal: -24.6276). This
parameter is required for the retrieval of GDAS data.
\item {\sc latitude\_key}: FITS keyword name for {\sc latitude} (default:
``ESO TEL GEOLAT'').

\end{itemize}
{\bf\large\tt\#\# INSTRUMENTAL PARAMETERS:}
\begin{itemize}
\item {\sc slitw}: Slit width in arcsec. The provided value is only taken into
account if the ESO keyword ``ESO INS SLIT1 WID'' does not exist and the
instrument is not X-Shooter (special keyword finding routine). For CRIRES and
VISIR, the stated keyword is usually present.
\item {\sc slitw\_key}: FITS keyword name for {\sc slitw} (default:
``ESO INS SLIT1 WID'').
\item {\sc pixsc}: Pixel scale in arcsec. This parameter has to be provided
manually, since this information could not be found in the ESO file headers of
the investigated instruments.
\item {\sc pixsc\_key}: FITS keyword name for {\sc pixsc}. The default is
``NONE'', \ie\ the parameter is read from the parameter file.
\end{itemize}

{\bf\large\tt\#\# ATMOSPHERIC PROFILES:}
\begin{itemize}
\item {\sc ref\_atm}: Reference atmospheric profile (standard profile). By
default, it is set to ``equ.atm'', which is located in the dedicated folder
``data/profiles/mipas/''. See Section~\ref{sec:mipas} for more information.
\item {\sc gdas\_prof}: Specific GDAS-like input profile (format:
P[hPa] HGT[m] T[K] RELHUM[\%], see Section~\ref{sec:gdas}). The path to the
file has to be relative to the installation directory or absolute. In the case of ``none'',
no GDAS profiles will be considered. The default option ``auto'' causes an
automatic retrieval either from a local library or a web-server (see
Section~\ref{sec:processing}).
\item {\sc layers}: Flag for grid of layer heights required for merging
{\sc ref\_atm} and GDAS profiles. The default option is 1, which selects a
fixed grid with 50 layers for Cerro Paranal (see Section~\ref{sec:processing}).
If the parameter is set to 0, a natural grid is used, \ie\ all layer heights of
{\sc ref\_atm} and the GDAS profile are combined, which tends to significantly
increase the number of layers compared to the default option. If local meteo
data are considered (see {\sc emix}), {\sc geoelev} is also added to the grid.
\item {\sc emix}: Upper mixing height in km for considering data of a local
meteorological station (see Section~\ref{sec:emm}). Above this level, the
influence of the meteo data is expected to be zero (see
Section~\ref{sec:processing}). For Cerro Paranal, the default value of 5\,km
is assumed. If {\sc emix} is below {\sc geoelev}, no local meteo data are
considered, \ie\ the content of {\sc rhum}, {\sc pres}, and {\sc temp} is
not used for modifying the merged profiles.
\item {\sc pwv}: Precipitable water vapour (PWV) value in mm for the input
water vapour profile. The merged profile composed of {\sc ref\_atm}, GDAS, and
local meteo data (see Section~\ref{sec:profiles}) will be scaled to this value
if {\sc pwv} is positive. By default ({\sc pwv}~=~-1), the manipulation of the
input water vapour profile is switched off. The {\sc relcol} value for water
vapour refers to the given {\sc pwv}.
\end{itemize}

The keyword {\sc end} is optional. It marks the last line that is considered by
the file reading routine. Any text beyond {\sc end} is ignored.

%-------------------------------------------------------------------------------
\subsection{Format of the input spectrum}\label{sec:inputspec}
%-------------------------------------------------------------------------------
\subsubsection{Accepted file formats}\label{sec:inputspec_accepted}
%-------------------------------------------------------------------------------
Molecfit currently accepts the following formats:
\begin{enumerate}
    \item ASCII.
    \item FITS binary table,
    \item 1D FITS image,
    \item Multi-dimentional FITS image,
\end{enumerate}
and recognizes it in the following way:
\begin{enumerate}[i]
    \item If the file is not a FITS file, Molecfit assumes an ASCII format.
    \item If the file is a FITS file, Molecfit then looks at the presence of
          extensions.  Molecfit assumes a 1D image format.  (Note that the FITS
          keyword {\sc EXTEND} only indicates that an extension may be present,
          not that it is present!)
    \item If there is at least  one extension, Molecfit uses the keyword
          {\sc XTENSION} in the extension(s) to identify if the file is
          a---possibly multi-extension---FITS binary table
          ({\sc XTENSION=BINTABLE}) or an image ({\sc XTENSION=IMAGE}).  If the
          file is an image, Molecfit then uses the {\sc NAXIS} keyword to
          distinguish between 1D, 2D and 3D images.
\end{enumerate}

%-------------------------------------------------------------------------------
\subsubsection{A note regarding the mask values}\label{sec:inputspec_mask}
%-------------------------------------------------------------------------------
The default behavior of Molecfit is that all values are either 0 or 1. A value
of 0 indicates that the value must not be used by the fit and a value of 1
indicates that the value should be used by the fit. However, Molecfit also
checks that values other than 0 and 1 are used. If this is the case, 0 is
assumed to be ok and all other values cause pixel rejection.

Possible {\sc NAN} in the flux and error columns are substituted by zero flux
and the corresponding pixel is rejected for the fit. The same is performed for
negative errors.


%-------------------------------------------------------------------------------
\subsubsection{Format of the ASCII files}\label{sec:inputspec_ascii}
%-------------------------------------------------------------------------------
Two columns are mandatory: the first column must provide the wavelength and the
second column, the flux. Two additional columns are optional: the 1-\(\sigma\)
error on the flux and the mask.

The 'columns' entry in the parameter file should be a list of 4 names, which are
unimportant, except for {\sc NULL}.  However, it is a good practice---if only
for readability---to use
\begin{verbatim}
	columns: Wavelength Flux Flux_Err Mask
\end{verbatim}

If the error on the flux and/or the mask are not provided, {\sc Flux\_Err}
and/or {\sc Mask} should be replaced by {\sc NULL}.

%-------------------------------------------------------------------------------
\subsubsection{Format of the FITS binary tables}\label{sec:inputspec_bintable}
%-------------------------------------------------------------------------------
Two columns are mandatory: the first column must provide the wavelength and the
second column the flux. Two additional columns are optional: the 1-\(\sigma\)
error on the flux and  the mask.

The 'columns' entry in the parameter file should be:
\begin{verbatim}
	columns: wavelength_label flux_label flux_err_label mask_label
\end{verbatim}
where the field is made of the label (title) of the column in the FITS binary
table. Note that the labels are case-sensitive! If the error on the flux and/or
the mask are not provided, {\sc flux\_err\_label} and/or {\sc mask\_label}
should be replaced by {\sc NULL}.

Example: reduced CRIRES (pre-upgrade) file using the optimally extracted fluxes.
\begin{verbatim}
	columns: Wavelength Extracted_OPT Error_OPT NULL
\end{verbatim}

%-------------------------------------------------------------------------------
\subsubsection{Format of the FITS 1D image data}\label{sec:inputspec_fits1dimg}
%-------------------------------------------------------------------------------
In the case of FITS image data, the wavelength information is provided through
the FITS keywords {\sc CRPIX1}, {\sc CRVAL1} and {\sc CRPIX1} if the wavelength
vector is not provided.

The 'columns' entry in the parameter file should be:
\begin{verbatim}
	columns: NULL Flux NULL NULL
\end{verbatim}

%-------------------------------------------------------------------------------
\subsubsection{Format of the FITS multi-extension image data}\label{sec_inputspec_fitsmultimg}
%-------------------------------------------------------------------------------
In the case of FITS image data, the wavelength information is provided through
the FITS keywords {\sc CRPIX1}, {\sc CRVAL1} and {\sc CRPIX1} if the wavelength
vector is not provided.

The name of the extension is retrieved from the {\sc EXTNAME} keyword of the
corresponding extension.

The 'columns' entry in the parameter file should be:
\begin{verbatim}
	columns: EXTNAME_for_wave EXTNAME_for_flux EXTNAME_for_error EXTNAME_for_mask
\end{verbatim}

If the error or the mask is not provided, the corresponding entry should be
replaced by {\sc NULL}.

Examples:
\begin{itemize}
	\item X-shooter visible spectrum reduced by the pipeline (esorex, gasgano,
          reflex), with the FITS keyword {\sc PRO.CATG} including any of the
          MERGE1D string.
\begin{verbatim}
	columns: NULL FLUX ERRS QUAL
\end{verbatim}
	\item X-shooter Internal Data Product spectrum (e.g. retrieved from the
          Phase 3 archive):
\begin{verbatim}
	columns: WAVE FLUX ERR QUAL
\end{verbatim}
\end{itemize}

%-------------------------------------------------------------------------------
\subsection{The output files}\label{sec:output}
%-------------------------------------------------------------------------------
\subsubsection{Output overview}\label{sec:outputfiles}
%-------------------------------------------------------------------------------
\sloppy The output files produced by {\tt molecfit}, {\tt calctrans},
{\tt calctrans\_lblrtm}, {\tt calctrans\_convolution} and
{\tt corrfilelist} are stored in the directory specified by the
{\sc output\_dir} parameter (see Section~\ref{sec:params}). The following
output files (named corresponding to the {\sc output\_name} parameter) are
created by {\tt molecfit}:
\begin{itemize}
\item <{\sc output\_name}>.fits: input file converted to FITS table and with
additional mask colum (1 = selected, 0 = rejected). This file can also be
produced by {\tt preptable}.
\item <{\sc output\_name}>\_fit.par: copy of input parameter file.
\item <{\sc output\_name}>\_fit.atm: final atmospheric profile for the best fit
(incorporates standard profile + \ac{GDAS} + \ac{EMM} + fit of the
molecular abundances).
\item <{\sc output\_name}>\_fit.fits: FITS table containing the observed input
spectrum and the modelled spectrum. There are 10 to 11 columns:
\#1: chip, \#2: wavelength grid of input spectrum converted into $\mu$m and
vacuum, \#3: observed input spectrum (radiance/ transmission), \#4: weight of
observed data, \#5: number of fit range (0 if not fitted), \#6: model
wavelengths in $\mu$m and vacuum, \#7: continuum scaling function,
\#8: modelled output spectrum (radiance/transmission), \#9: weight of model
pixels (0 = no valid calculation), \#10: weighted deviation between modelled
and observed spectrum, \#11: transmission curve for fitted wavelength ranges
(only for transmission case).
\item <{\sc output\_name}>\_fit.asc: ASCII version of the FITS table described
above.
\item <{\sc output\_name}>\_fit.ps: (optional) postscript plot showing a
comparison of the best-fit model and input observed spectrum and the difference
of both spectra. See parameter {\sc plot\_creation}.
\item <{\sc output\_name}>\_fit\_<fitrange>.ps: (optional) postscript plots
for the individual fit ranges. See the parameters {\sc plot\_creation} and
{\sc plot\_range}.
\item <{\sc output\_name}>.res: results file containing information on the fit
quality and the best-fit parameters (see Section~\ref{sec:resfile}).
\end{itemize}
The executable {\tt calctrans} creates the following files:
\begin{itemize}
\item <{\sc output\_name}>\_tac.fits: FITS table containing the results of
the telluric absorption correction of the input spectrum. There are 9 columns:
\#1: chip, \#2: wavelength grid of input spectrum converted into $\mu$m and
vacuum, \#3: observed input spectrum (radiance/transmission), \#4: weight of
observed data, \#5: model wavelengths in $\mu$m and vacuum, \#6: best-fit
transmission curve (correction function for telluric absorption),
\#7: weight of model pixels (0 = no valid calculation), \#8: input spectrum
corrected for telluric absorption, \#9: flag indicating the quality of the
telluric absorption correction (0 = very low transmission $\to$ zero-point
uncertainties are crucial $\to$ numerical problems expected, 1 = probably OK).
A model transmission curve (column \#6) is also calculated for input sky
radiance spectra. However, in this case, no telluric absorption correction
is carried out (column \#8).
\item <{\sc output\_name}>\_tac.asc: ASCII version of the FITS table described
above.
\item <{\sc output\_name}>\_tac.ps: (optional) postscript plot showing a
comparison of the input spectrum with and without telluric absorption
correction. See parameter {\sc plot\_creation}.
\item <{\sc filename}>\_TAC.<filetype>: telluric absorption corrected input
spectrum in the same file format as the input file. For ASCII and FITS tables,
a corresponding column plus the quality flag column are added. If an error
column exists, a column for the corrected error is also created. For FITS
images, the flux spectrum is substituted by the corrected spectrum and
existing flux error and mask/quality FITS extensions are modified.
\item <{\sc filename}>\_TRA.<filetype>: 1D FITS image without extensions that
holds the correction function for telluric absorption. This file is only
created if the input file is a FITS image.
\end{itemize}
Alternatively, the executable {\tt calctrans\_lblrtm} creates the following
files:
\begin{itemize}
\item <{\sc output\_name}>\_N\_T|R.fits: FITS table containing the result of
    the telluric absorption features where:
    \begin{itemize}
    \item N is a number corresponding to the fit range,
    \item T or R is appended to the filename according to the user's
        request: transmission or emission spectrum ({\tt trans} keyword in
        the input parameter file).
    \end{itemize}
\item <{\sc output\_name}>\_code\_stat.flags: flags used internally between
    \sloppy {\tt calctrans\_lblrtm} and {\tt calctrans\_convolution}.
\end{itemize}
The executable {\tt calctrans\_convolution} creates the following files:
\begin{itemize}
\item <{\sc output\_name}>\_tac.fits
\item <{\sc output\_name}>\_tac.asc
\item <{\sc output\_name}>\_tac.ps
\item <{\sc filename}>\_TAC.<filetype>
\item <{\sc filename}>\_TRA.<filetype>
\end{itemize}
(see the paragraph above about {\tt calctrans} for a detailed description of
those files)

Finally, {\tt corrfilelist} produces a <filename>\_TAC.<filetype> file (see
above) for each file listed in {\sc listname}.

%-------------------------------------------------------------------------------
\subsubsection{Example of a {\tt .res} file}\label{sec:resfile}
%-------------------------------------------------------------------------------
The <{\sc output\_name}>\_fit.res file contains detailed information on the fit
results. In particular, information on the fit quality, i.e. $\chi^2$ and RMS
values, all coefficients of the best-fit model, uncertainties of these
coefficients (only if a parameter was fitted), and the final water column are
given. For the provided {\tt mpfit} status message, see the documentation in
{\tt mpfit.h}. In general, positive numbers imply that the code found a
solution.

In the following, the output file belonging to the parameter file listed in
Section~\ref{sec:paramfile} is shown:
\begin{verbatim}
DATA FILE:
examples/input/crires_spec_jitter_extracted_0000.fits

MPFIT RESULTS:
Status:                    2
Fit parameters:            38
Data points:               4096
Weight > 0:                3896
Frac. of valid model pix.: 1.00
Iterations:                13
Function evaluations:      154
Fit run time in min:       1.15
Avg. LBLRTM run time in s: 3.76
LBLRTM calls:              16
Initial chi2:              1.452e+09
Best chi2:                 4.562e+05
Reduced chi2:              1.171e+02
RMS rel. to error:         1.082e+01
RMS rel. to mean:          1.990e-02

BEST-FIT PARAMETERS:

SPECTRAL RESOLUTION:
Rel. FWHM of boxcar (slit width = 1): 0.000
FWHM of boxcar in pixels:             0.000
FWHM of Gaussian in pixels:           1.940 +- 0.003
FWHM of Lorentzian in pixels:         0.500

WAVELENGTH SOLUTION:
Chip 1, coef 0: -5.082e-03 +- 4.356e-06
Chip 1, coef 1:  1.001e+00 +- 7.231e-06
Chip 1, coef 2:  2.085e-03 +- 7.277e-06
Chip 1, coef 3:  5.023e-04 +- 6.830e-06
Chip 2, coef 0:  2.314e-03 +- 4.339e-06
Chip 2, coef 1:  1.004e+00 +- 9.810e-06
Chip 2, coef 2:  5.840e-03 +- 6.325e-06
Chip 2, coef 3:  4.218e-05 +- 6.381e-06
Chip 3, coef 0: -3.487e-04 +- 7.627e-06
Chip 3, coef 1:  9.995e-01 +- 1.836e-05
Chip 3, coef 2:  1.551e-03 +- 9.098e-06
Chip 3, coef 3: -7.590e-05 +- 8.770e-06
Chip 4, coef 0: -1.360e-03 +- 3.974e-06
Chip 4, coef 1:  1.001e+00 +- 7.622e-06
Chip 4, coef 2: -1.099e-03 +- 6.119e-06
Chip 4, coef 3:  1.735e-05 +- 6.554e-06

CONTINUUM CORRECTION:
Range 1, chip 1, coef 0:  4.395e+01 +- 3.220e-03
Range 1, chip 1, coef 1:  6.602e+00 +- 1.070e+00
Range 1, chip 1, coef 2: -8.195e+03 +- 9.750e+01
Range 1, chip 1, coef 3: -5.589e+05 +- 2.507e+04
Range 2, chip 2, coef 0:  4.375e+01 +- 3.903e-03
Range 2, chip 2, coef 1:  1.293e+02 +- 1.375e+00
Range 2, chip 2, coef 2: -3.498e+04 +- 1.412e+02
Range 2, chip 2, coef 3: -2.385e+06 +- 3.431e+04
Range 3, chip 3, coef 0:  3.624e+01 +- 7.714e-03
Range 3, chip 3, coef 1: -1.796e+02 +- 2.219e+00
Range 3, chip 3, coef 2: -5.348e+03 +- 6.280e+02
Range 3, chip 3, coef 3:  2.646e+06 +- 8.653e+04
Range 4, chip 4, coef 0:  3.245e+01 +- 3.088e-03
Range 4, chip 4, coef 1:  5.918e+01 +- 1.094e+00
Range 4, chip 4, coef 2: -3.674e+03 +- 1.265e+02
Range 4, chip 4, coef 3:  2.851e+05 +- 3.342e+04

RELATIVE MOLECULAR GAS COLUMNS:
H2O: 0.885 +- 0.000
CH4: 0.946 +- 0.000
 O3: 0.932 +- 0.001

MOLECULAR GAS COLUMNS IN PPMV:
H2O: 2.051e+02 +- 4.099e-02
CH4: 1.639e+00 +- 3.884e-04
 O3: 3.909e-01 +- 4.666e-04

H2O COLUMN IN MM: 0.991 +- 0.000
\end{verbatim}

%-------------------------------------------------------------------------------
\subsection{Known Issues}\label{sec:issues}
%-------------------------------------------------------------------------------
\begin{itemize}
  \item If the data have pixels with unusually high values, Molecfit may fail since the
        underlying algorithm uses single precision floats.  A solution is to apply a
        scale factor to the data before the fit.
  \item The old MacOSX file system (HDF+) have case insensitivity. The user must be 
        sure to put a different name in the 'output\_name' parameter that the name of
        the input file in Molecfit. If you don't take care of that the GUI will not
        show any error message in the screen in the old MacOSX file systems, 
        but it will not storage the output correction file.  
\end{itemize}
