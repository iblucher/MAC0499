%-------------------------------------------------------------------------------
\section{Installation procedure}\label{sec:installation}
%-------------------------------------------------------------------------------
%-------------------------------------------------------------------------------
\subsection{Requirements}\label{sec:requirements}
%-------------------------------------------------------------------------------
The installation of the basic \mf{} binary package requires:
\begin{itemize}
  \item
    C99 compatible compiler (e.g. gcc or clang)
  \item
    glibc 2.11 or newer on Linux or OS X 10.7 or newer
  \item
    common unix utilities (bash, tar, sed, grep, \ldots{})
\end{itemize}

The optional \ac{GUI} to \mf{} requires:

\begin{itemize}
  \item
    Python v2.6 or v2.7 (but not Python v3.x)
  \item
    wxPython v2.8 or newer
  \item
    Python matplotlib v1.0 or newer
  \item
    PyFITS v2.4 or newer
\end{itemize}

The command line client also has optional display features which
require:

\begin{itemize}
  \item gnuplot v4.2 patchlevel 3 or newer
\end{itemize}

%-------------------------------------------------------------------------------
\subsection{Binary installation}
\label{sec:installscript}
%-------------------------------------------------------------------------------

First the downloaded installer needs to be made executable. To do this change
into the directory the installer was downloaded to and run following command
(replacing molecfit\_installer.run with the actual downloaded filename):

\begin{verbatim}
chmod u+x ./molecfit_installer.run
\end{verbatim}

Now the installer can be executed from the same folder with:

\begin{verbatim}
./molecfit_installer.run
\end{verbatim}

It will ask for an installation directory where it will extract its
contents to. It is recommended to choose an empty directory to avoid
overwriting existing files.

After the installer has successfully finished, the \mf{} executables
are installed into the \texttt{bin} subdirectory of the chosen installation
folder. They can be executed by specifying the full or relative path.
Also installed are a set of example parameter files for several
instruments in the \texttt{examples/config} directory. To run a CRIRES example
type:

\begin{verbatim}
<INST_DIR>/bin/molecfit <INST_DIR>/examples/config/molecfit_crires.par
\end{verbatim}

For more details see Section~\ref{sec:running}.

For the \ac{GUI} use the \texttt{molecfit\_gui} executable instead of
\texttt{molecfit}.
For more details on the \ac{GUI} see \cite{MFGUI}.

%-------------------------------------------------------------------------------
\subsection{\ac{GUI} dependencies}
\label{sec:guidependencies}
%-------------------------------------------------------------------------------

The \ac{GUI} requires some additional dependencies to be installed on the system.
To check if the python installation is able to run the \ac{GUI}, following
commands can be run:

\begin{verbatim}
python -c 'import wx'

python -c 'import matplotlib; import matplotlib.backends.backend_wxagg'

python -c 'import pyfits'
\end{verbatim}

If these commands fail please see following site for instructions on how
to install these packages:

\texttt{http://www.eso.org/pipelines/reflex\_workflows/}

%-------------------------------------------------------------------------------
\subsection{Package contents}
\label{sec:pkgcontents}
%-------------------------------------------------------------------------------

The installation package is a self extracting tarball containing the
\mf{} source code and pre-built versions of its third party
dependencies:

\begin{itemize}
  \item
    Common Pipeline Library v6.4.2 and its dependencies cfitsio v3.350,
    wcslib v4.16 and fftw3 v3.3.3 \cite{CPL}
  \item
    Gridded Binary (GRIB) \gribv{} \cite{WGRIB}
  \item
    radiative transfer code \lnflv{} and \lblrtmv{} \cite{LBLRTM}
  \item
    \aerv{} \cite{AER}
\end{itemize}


%-------------------------------------------------------------------------------
\subsection{Source Installation}
\label{sec:sourceinstall}
%-------------------------------------------------------------------------------

Advanced users may want to install everything from source, the basic
instructions for this are outlined here. The installation from source
additionally requires the gfortran compiler.

\subsubsection{CPL compilation}

The CPL sources can be obtained from \cite{CPL}

For \mf{} CPL only requires cfitsio. It can be installed as follows:

\begin{verbatim}
./configure --prefix=/install-location

make

make shared

make install
\end{verbatim}

Then CPL can be install with:

\begin{verbatim}
./configure --prefix=/install-location --with-cfitsio=/install-location

make

make install
\end{verbatim}

See the respective packages documentation for details on the
installation procedure.

\subsubsection{Non-ESO sources compilation}

The rest of the required sources can be downloaded from the same place
the \mf{} binaries installers are available from. The third party
source tarball contains a convenience Makefile to build and install all
packages at once. It can be run with:

\begin{verbatim}
make -f BuildThirdParty.mk install prefix=/install-location
\end{verbatim}

For installation on Mac OS please read the comments in this Makefile.

\subsubsection{Molecfit compilation}

After all dependencies have been installed \mf{} can be compiled from
source into the same location.

This is the only step required if one wants to update \mf{} from
source after previously installing the third party dependencies with the
binary installer.

\begin{verbatim}
./configure --prefix=/install-location --with-cpl=/install-location

make

make install
\end{verbatim}

In order to use \mf{} from this location the environment variable
{\tt LD\_LIBRARY\_PATH} \\(or {\tt DYLD\_LIBRARY\_PATH} on Mac OS) need to be
set. With the bash shell this is done with following command:

\begin{verbatim}
export LD_LIBRARY_PATH=/install-location/lib
\end{verbatim}

Now \mf{} is ready to be used from {\tt /install-location/bin}.
