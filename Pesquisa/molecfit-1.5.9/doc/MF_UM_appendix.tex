%-------------------------------------------------------------------------------
\section{Appendix}\label{app:appendix}
%-------------------------------------------------------------------------------
\subsection{Expert Fitting}\label{app:expertfit}
%-------------------------------------------------------------------------------
The expert fitting mode allows one to fit individual chips and ranges and
provides access to the coefficients of the polynomials for the continuum and
wavelength correction.
Moreover, parameter files with the best-fit values are written
which can be used for another iteration of \mf.

The example parameter file {\tt molecfit\_crires\_expert.par} can be used to
test the expert mode. The following rows differ from the standard parameter
file:

% \begin{verbatim}
% # Maximum number of chips to be checked for fit_chip and cont_chip
% # keywords (default: 4)
% nchip_max
% \end{verbatim}

% The parameters {\tt nchip\_max} is only required if there are more
% than 4 chips. Since this is not the case for the given CRIRES
% example, they can be ignored.

\begin{verbatim}
# Fit flag for fitting ranges with respect to continuum correction
# (default: fit_cont)
fit_range1:
fit_range2:
fit_range3:
fit_range4:
\end{verbatim}

The parameters {\tt fit\_range[range]} allow one to enter fit flags for the
different fitting ranges with respect to the continuum correction. These
parameters overrule the {\tt fit\_cont} parameter if set. Note that a change of
the fitting ranges can change the numbering. User-defined ranges that extend
over more than one chip will be split and numbered differently.

\begin{verbatim}
# Fit flag for chips with respect to wavelength correction
# (default: fit_wlc)
fit_chip1:
fit_chip2:
fit_chip3:
fit_chip4:
\end{verbatim}

The parameters {\tt fit\_chip[chip]} are very similar to {\tt fit\_range[range]}.
They allow for independent fits of the wavelength correction coefficients for
each chip. They overrule {\tt fit\_wlc} if set.

\begin{verbatim}
# Range-specific coefficients for continuum correction (list separated by
# spaces; default: cont_const 0. 0. ...)
cont_range1:
cont_range2:
cont_range3:
cont_range4:
\end{verbatim}

The parameters {\tt cont\_range[range]} contain the coefficients for the
continuum correction for each fitting range. They have to be listed separated
by spaces.
The number of coefficients per parameter is not restricted. They can deviate
for different ranges. However, note that the parameter {\tt cont\_n} sets a limit to
the number of coefficients that are considered for the fit ({\tt cont\_n + 1}).
The set coefficients overrule the standard values, which are
{\tt cont\_const, 0., 0.}, and so on.

\begin{verbatim}
# Chip-specific coefficients for wavelength correction (list separated by
# spaces; default: wlc_const 1. 0. 0. ...)
wlc_chip1:
wlc_chip2:
wlc_chip3:
wlc_chip4:
\end{verbatim}

The parameters {\tt wlc\_chip[range]} contain the coefficients for the
wavelength correction for each chip. The maximum number of coefficients taken
for the fit is {\tt wlc\_n + 1}.
The set coefficients overrule the standard values, which are
{\tt wlc\_const, 1., 0., 0.}, and so on.

Each provided fit flag or coefficient for the continuum and wavelength
correction will be considered as start value for the fitting procedure.
If no value is provided, \mf{} runs in the standard mode, i.e. the default
values are taken. There is a difference in the performance if all colons are
removed from the parameter labels. In this case, the traditional \mf{} mode
runs as described in the user manual. However, a single colon will already
activate the expert mode. As a consequence, an additional output file will
be written to the output directory. It has the ending {\tt \_fit.rpar}.
It equals the file for the \mf{} input parameters labelled {\tt \_fit.par}
except for the fit parameters, where the best-fit value is written.  Apart from
the {\tt cont\_range} and {\tt wlc\_chip} parameters, these are {\tt relcol},
{\tt telback}, {\tt relres\_box}, {\tt res\_gauss}, and {\tt res\_lorentz}.
The advantage of this file is that it can directly be used as input for another
run of \mf. By modifying the fit flags (and parameter values) in this file
before calling \mf, the fit results can be improved by focusing on subsets of
the fitting parameters, which makes the procedure more robust.

The expert mode offers full access to the \mf{} fitting parameters and
allows for more flexible iterative fitting. On the other hand, it increases
the complexity of the fitting procedure and there is a chance of erroneous
results if the mask files are changed and the range and chip-related parameters
are not carefully adapted. For this reason, it has not been implemented as
standard mode for \mf.

%-------------------------------------------------------------------------------
\subsection{Maintenance}\label{app:maintenance}
%-------------------------------------------------------------------------------
\subsubsection{Introduction}
%-------------------------------------------------------------------------------
The package \mf{} relies on some external codes and data, which might undergo a
change due to development even after the SM-02 project ends. Although, it is
not possible to foresee all possible modifications, some hints can be given to
use \mf{} even with updates from the external sources at a later stage. This
particularly applies to the radiative transfer code LNFL/LBLRTM, and the GDAS
data. In this appendix, we summarise the issues to be taken into account for a
successful usage of \mf{} with later versions of LNFL/LBLRTM, the line
database, and the GDAS data.

%-------------------------------------------------------------------------------
\subsubsection{Radiative transfer code LNFL/LBLRTM}\label{app:lblrtm}
%-------------------------------------------------------------------------------
The radiative transfer code package LNFL/LBLRTM is developed by
AER\cite{LBLRTM} and can directly be obtained from there. \mf{} is delivered
with \lnflv, \lblrtmv, and the \aerv.

In principle, \mf{} should also work with later versions as its principle usage
has been unchanged for several years. This means that \mf{} is expected to
provide all functionality as long as it is retained. In particular, the
principle usage of the TAPE<xx> files \cite{LBLRTMFAQ}, and the format of the
LBLRTM input file {\tt TAPE5} MUST remain unchanged. However, as LBLRTM is
widely used in atmospheric research, we do not expect major changes in the near
future. In the course of the former DR-, and later SM-projects, we used several
versions without problems.

What might change is the installation procedure of the LNFL/LBLRTM codes. If a
future release of \mf{} is provided with a newer version of the radiative
transfer code, the install script might need to be modified accordingly. For
that purpose, the developers from AER provide a README file in the tree of the
sources. In order to maximise flexibility, \mf{} only searches for executables
labelled {\tt lnfl} and {\tt lblrtm} in the {\tt <INST\_DIR>/bin/}
directory. Therefore, a soft link to the actual binaries of LNFL and LBLRTM has
to be provided by the installation script. To facilitate the modification in
the script, the corresponding section is marked there.

We recommend the following procedure for the update of \mf{} to later versions
of LNFL/LBLRTM:
\begin{itemize}
    \item Updates of the LNFL/LBLRTM are usually once or twice per year (see
Section "What's new" at \cite{LBLRTM}). Therefore, a quarterly check for new
versions is sufficient.
    \item Download the latest version of LNFL/LBLRTM from \cite{LBLRTM}
    \item Make sure that the principle usage has not changed. Check the
documents in the {\tt docs} directory in the tree of the LBLRTM sources. There,
you should find a FAQ document, the release notes, and a detailed HTML
description of the {\tt TAPE5} input file for LBLRTM.
    \item Read the {\tt README} files whether the installation procedure has
changed. These files should be located in the {\tt src/build} directory of LNFL
and LBLRTM.
    \item Check whether there is also a line list update and follow the steps
described in Section~\ref{app:linelist}.
    \item Adapt the installation script of \mf{}.
    \item Evaluate by means of the delivered examples.
\end{itemize}

%-------------------------------------------------------------------------------
\subsubsection{AER Line parameter list}\label{app:linelist}
%-------------------------------------------------------------------------------
Usually the line parameter list delivered with the LNFL/LBLRTM package is
updated at the same time as the code. We recommend to use the version which is
delivered together with the corresponding code version.

To integrate the new line list, simply unpack the tarball and copy the new line
parameter file (usually labelled {\tt aer\_v\_<version>} to the directory {\tt
<INST\_DIR>/data/hitran/} of your \mf{} installation. This can be
done also by modifying the install script accordingly. The corresponding
section is marked there in a special way. Moreover, it is necessary to provide
the new line list name to the {\sc \_line\_db} parameter in the
{\tt lblrtm\_setup} file in the {\tt<INST\_DIR>/config/} directory.

Note that it has turned out that at long wavelengths relevant for VISIR, \mf{}
can be significantly slower if the recommended {\tt aer} line list is used
instead of the basic HITRAN~2008 database (see Section~\ref{sec:pwv}).
Unfortunately, this line list is no more available at the HITRAN website
\cite{HITRAN}, which now provides the 2012 version. The new version has not
been tested in terms of performance in \mf{}. Hence, it cannot be guaranteed
that it works at all. If you want to test an original HITRAN line list, you
have to put it in the {\tt data/hitran/} folder and modify {\sc \_line\_db} in
the {\tt lblrtm\_setup} file. In addition, the parameter {\sc \_line\_db\_fmt}
has to be changed from 100 to 160 because of a different file format.

%-------------------------------------------------------------------------------
\subsubsection{GDAS data}\label{app:gdas}
%-------------------------------------------------------------------------------
In contrast to the radiative transfer code, which is updated once/twice per
year, the GDAS profiles are updated permanently. \mf{} initially searches in a
local tarball for the corresponding profiles (script \\{\tt
<INST\_DIR>/bin/get\_gdas\_profiles.sh}), and tries to download it
from the web in the case they are not found locally. The latter is done by the
script {\tt <INST\_DIR>/bin/extract\_grib.sh} with the help of the
GRIB software.

However, the incorporated download server\footnote{\tt
http://nomad3.ncep.noaa.gov/pub/gdas/rotating/} does not seem to be very stable
and organised in a predictable way. It might happen, that the profiles are
not found.
Therefore regularly updated data tarballs are available from \molecpage.
They can be downloaded and placed into the {\tt <INST\_DIR>/data/profiles/gdas}
folder in case the automatic update mechanism fails.

Additionally \mf{} contains the script used to produce the updated files for
the website. It can also be used to update the local GDAS database. The script
downloads the GDAS data archive from a different
server\footnote{\tt ftp://ftp.arl.noaa.gov/archives/gdas1/}. The data there
is organised on weekly basis, each stored in a single file labelled
{\tt gdas1.<mmm><yy>.w\#}, being <mmm> the month (e.g. jan, feb, mar,...) and
<yy> the year. The extension "w\#" describes the number of the week of this
month:

\#=1 - days 1-7 of the month\\
\#=2 - days 8-14\\
\#=3 - days 15-21\\
\#=4 - days 22-28\\
\#=5 - days 29 - rest of the month \\

For example, the file {\tt gdas1.apr07.w3} contains data of the third
week of April 2007 (15th to 21st of April 2007). More information on the file
structure is given here\footnote{\tt http://www.ready.noaa.gov/gdas1.php}.

These files provide the GDAS profiles of the world wide grid. Therefore, it is
necessary to extract the profiles in the same way as described in
Section~\ref{sec:profiles}. This can be achieved with the shell script {\tt
update\_gdas\_db.sh}. It downloads the data on monthly basis, and extracts the
profiles as required by \mf{}, and adds them to the local database. However,
this may take a while as the amount of data to be downloaded is large (up to
580\,MB per week) and the server is fairly slow.

The shell script invokes a C-programme {\tt extract\_gdas\_profiles}. Although,
both tools are part of the molecfit package, they can be used independently of
\mf{} to provide the possibility to update the GDAS archive on different
machines. This might be useful as some disc space is required temporarily.  If
this is intended, both tools have to be installed in the following way on the
target machine:

\begin{itemize}
\item
  create subfolders {\tt bin/} and {\tt data/downloads/paranal/} in the target
  directory on the target machine.
\item
  Copy the shell script {\tt update\_gdas\_db.sh} and the C source file
  {\tt extract\_gdas\_profiles.c} to the {\tt bin/} folder in target directory
  on the target machine.
\item
  There compile the C programme by invoking
\begin{verbatim}
    gcc extract_gdas_profiles.c -lm -o extract_gdas_profiles
\end{verbatim}
\item
  Copy the tarball file {\tt gdas\_profiles\_C-70.4-24.6.tar.gz} delivered with
  the \mf{} package to the directory
  {\tt <targetdir@targetmachine>/data/downloads/paranal/}. This tarball is
  being updated.
\end{itemize}

To update the local GDAS database please follow these steps:
\begin{itemize}
    \item
The final release of \mf{} contains GDAS profiles up to March 2014.
So, if this is your first update, invoke the update script with the
parameter oct13:
\begin{verbatim}
    cd <INST_DIR>/
\end{verbatim}
OR
\begin{verbatim}
    cd <targetdir@targetmachine>
\end{verbatim}
(depending on your installation (see above)
\item invoke the update script
\begin{verbatim}
    bin/update_gdas_database.sh oct13 P
\end{verbatim}
and proceed until the end of the last fully missing month, if necessary. Note:
You only can download full months. The file
{\tt data/paranal/last\_month.txt} shows a history of past updates.
\item Copy the resulting tarball\\
{\tt data/downloads/paranal/gdas\_profiles\_C-70.4-24.6.tar.gz} to the folder
\\{\tt <INST\_DIR>/data/profiles/gdas/} in your \mf{} installation path.
\item Optionally: delete the downloaded files in {\tt data/downloads/gdas/}.
\end{itemize}
Some dates are missing. Please check the website\footnote{\tt
http://ready.arl.noaa.gov/archives.php} for more details. There are no GDAS
data available before 1st of December 2004.

%-------------------------------------------------------------------------------
\subsection{License issues}\label{app:licenses}
%-------------------------------------------------------------------------------
\subsubsection{LNFL/LBLRTM}
%-------------------------------------------------------------------------------
{\bf URL:} http://rtweb.aer.com/lblrtm\_frame.html

\paragraph*{LNFL Terms and Conditions}

The LNFL software you are downloading has been copyrighted by Atmospheric and
Environmental Research, Inc. (AER) and is proprietary to AER. AER grants USER
the right to download, install, use and copy this software for scientific and
research purposes only. By downloading this software, you are agreeing that:
\begin{itemize}
\item If you redistribute any portion of the software, whether in original or
modified form, each source code file redistributed will include the AER
copyright notice currently present in that file.
\item The software or any modified version of the software may not be
incorporated into proprietary software or commercial software offered for sale.
\item Should you utilize the results of LNFL in your research, we request that
you appropriately reference this usage in any associated publications.
\item This software is provided as is without any express or implied
warranties.
\end{itemize}

\paragraph*{LBLRTM Terms and Conditions}

The LBLRTM software you are downloading has been copyrighted by Atmospheric and
Environmental Research, Inc. (AER) and is proprietary to AER. AER grants USER
the right to download, install, use and copy this software for scientific and
research purposes only. By downloading this software, you are agreeing that:
\begin{itemize}
\item If you redistribute any portion of the software, whether in original or
modified form, each source code file redistributed will include the AER
copyright notice currently present in that file.
\item The software or any modified version of the software may not be
incorporated into proprietary software or commercial software offered for sale.
\item Should you utilize the results of LBLRTM in your research, we request
that you appropriately reference this usage in any associated publications.
\item This software is provided as is without any express or implied
warranties.
\end{itemize}

\paragraph*{Principal References}

Clough, S. A., M. W. Shephard, E. J. Mlawer, J. S. Delamere, M. J. Iacono, K.
Cady-Pereira, S. Boukabara, and P. D. Brown, Atmospheric radiative transfer
modeling: a summary of the AER codes, Short Communication, J. Quant. Spectrosc.
Radiat. Transfer, 91, 233-244, 2005.

Clough, S.A., M.J. Iacono, and J.-L. Moncet, Line-by-line calculation of
atmospheric fluxes and cooling rates: Application to water vapor. J. Geophys.
Res., 97, 15761-15785, 1992.

%-------------------------------------------------------------------------------
%\subsubsection{RFM}
%-------------------------------------------------------------------------------
%{\bf  URL :} http://www.atm.ox.ac.uk/RFM/

%-------------------------------------------------------------------------------
\subsubsection{cmpfit library}
%-------------------------------------------------------------------------------
{\bf  URL :} http://www.physics.wisc.edu/~craigm/idl/cmpfit.html
{\bf  URL :} http://cow.physics.wisc.edu/~craigm/idl/idl.html

\paragraph*{DISCLAIMER}

MPFIT: A MINPACK-1 Least Squares Fitting Library in C

Original public domain version by B. Garbow, K. Hillstrom, J. More'
  (Argonne National Laboratory, MINPACK project, March 1980)
  Copyright (1999) University of Chicago
    (see below)

Tranlation to C Language by S. Moshier (moshier.net)
  (no restrictions placed on distribution)

Enhancements and packaging by C. Markwardt
  (comparable to IDL fitting routine MPFIT

   see http://cow.physics.wisc.edu/~craigm/idl/idl.html)
  Copyright (C) 2003, 2004, 2006, 2007 Craig B. Markwardt

  This software is provided as is without any warranty whatsoever.
  Permission to use, copy, modify, and distribute modified or
  unmodified copies is granted, provided this copyright and disclaimer
  are included unchanged.


Source code derived from MINPACK must have the following disclaimer
text provided.

===========================================================================\\
Minpack Copyright Notice (1999) University of Chicago.  All rights reserved

Redistribution and use in source and binary forms, with or
without modification, are permitted provided that the
following conditions are met:

1. Redistributions of source code must retain the above
copyright notice, this list of conditions and the following
disclaimer.

2. Redistributions in binary form must reproduce the above
copyright notice, this list of conditions and the following
disclaimer in the documentation and/or other materials
provided with the distribution.

3. The end-user documentation included with the
redistribution, if any, must include the following
acknowledgment:

   "This product includes software developed by the
   University of Chicago, as Operator of Argonne National
   Laboratory.

Alternately, this acknowledgment may appear in the software
itself, if and wherever such third-party acknowledgments
normally appear.

4. WARRANTY DISCLAIMER. THE SOFTWARE IS SUPPLIED "AS IS"
WITHOUT WARRANTY OF ANY KIND. THE COPYRIGHT HOLDER, THE
UNITED STATES, THE UNITED STATES DEPARTMENT OF ENERGY, AND
THEIR EMPLOYEES: (1) DISCLAIM ANY WARRANTIES, EXPRESS OR
IMPLIED, INCLUDING BUT NOT LIMITED TO ANY IMPLIED WARRANTIES
OF MERCHANTABILITY, FITNESS FOR A PARTICULAR PURPOSE, TITLE
OR NON-INFRINGEMENT, (2) DO NOT ASSUME ANY LEGAL LIABILITY
OR RESPONSIBILITY FOR THE ACCURACY, COMPLETENESS, OR
USEFULNESS OF THE SOFTWARE, (3) DO NOT REPRESENT THAT USE OF
THE SOFTWARE WOULD NOT INFRINGE PRIVATELY OWNED RIGHTS, (4)
DO NOT WARRANT THAT THE SOFTWARE WILL FUNCTION
UNINTERRUPTED, THAT IT IS ERROR-FREE OR THAT ANY ERRORS WILL
BE CORRECTED.

5. LIMITATION OF LIABILITY. IN NO EVENT WILL THE COPYRIGHT
HOLDER, THE UNITED STATES, THE UNITED STATES DEPARTMENT OF
ENERGY, OR THEIR EMPLOYEES: BE LIABLE FOR ANY INDIRECT,
INCIDENTAL, CONSEQUENTIAL, SPECIAL OR PUNITIVE DAMAGES OF
ANY KIND OR NATURE, INCLUDING BUT NOT LIMITED TO LOSS OF
PROFITS OR LOSS OF DATA, FOR ANY REASON WHATSOEVER, WHETHER
SUCH LIABILITY IS ASSERTED ON THE BASIS OF CONTRACT, TORT
(INCLUDING NEGLIGENCE OR STRICT LIABILITY), OR OTHERWISE,
EVEN IF ANY OF SAID PARTIES HAS BEEN WARNED OF THE
POSSIBILITY OF SUCH LOSS OR DAMAGES.

%-------------------------------------------------------------------------------
\subsubsection{HITRAN}
%-------------------------------------------------------------------------------
{\bf URL:} http://www.cfa.harvard.edu/hitran/\\
{\bf URL}: http://www.cfa.harvard.edu/hitran/Updated/ref-table.pdf

\paragraph*{Principal references}

Rothman et al., "The HITRAN 2008 molecular spectroscopic database", Journal of
Quantitative Spectroscopy and Radiative Transfer, vol. 110, pp. 533-572 (2009)

%-------------------------------------------------------------------------------
\subsubsection{GDAS}
%-------------------------------------------------------------------------------
{\bf URL:} http://ready.arl.noaa.gov/gdas1.php

\paragraph*{Air Resources Laboratory - Disclaimer}

The information on government servers is in the public domain, unless
specifically annotated otherwise, and may be used freely by the public. Before
using information obtained from this server special attention should be given
to the date and time of the data and products being displayed. This information
shall not be modified in content and then presented as official government
material. The user shall credit NOAA Air Resources Laboratory in any
publications, presentations, or other derivative works that result from the use
of this material. NOAA Air Resources Laboratory is not obligated to provide any
support, consulting, training, or assistance of any kind with regard to the use
of this material.

This material is provided "AS IS" and any express or implied warranties
including, but not limited to, the warranties of merchantability and fitness
for a particular purpose are disclaimed. In no event shall NOAA Air Resources
Laboratory be liable to you or to any third party for any direct, indirect,
incidental, consequential, special or exemplary damages or lost profit
resulting from any use or misuse of this data.

This server is available 24 hours a day, seven days a week, however this server
is not operational and is not maintained 24 hours a day, seven days a week.
Timely delivery of data and products from this server through the Internet is
not guaranteed.

Any references or links from the ARL web server to any non-Federal Government
Web sites do not imply endorsement of any particular product, service,
organization, company, information provider, or content. We are not responsible
for the contents of any "off-site" web pages referenced from ARL servers.

%-------------------------------------------------------------------------------
%\subsubsection{ESO Meteo Monitor / HATPRO radiometer data}
%-------------------------------------------------------------------------------

%-------------------------------------------------------------------------------
%\subsection{Acknowledgements}
%-------------------------------------------------------------------------------
%We would like to thank.....***TBD***
