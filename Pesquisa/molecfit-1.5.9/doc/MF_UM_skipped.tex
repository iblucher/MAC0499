
%-------------------------------------------------------------------------------
\subsection{The manual way}\label{sec:manualway}
%-------------------------------------------------------------------------------
If anything goes wrong with the installation script the package can be
installed manually, step by step, except the Reflex workflow.

%-------------------------------------------------------------------------------
\subsubsection{Building {\tt molecfit}}\label{sec:building}
%-------------------------------------------------------------------------------
The software expects a directory structure as provided in the tarball:
\begin{verbatim}
               |--bin/
               |
  <basedir>----|--config/
               |
               |--data/
\end{verbatim}
where {\tt <basedir>} is the installation directory.
% The names of the subdirectories are fixed except for the output directory, for
% which a different name can be set in the parameter file of {\tt molecfit}
% (see Section~\ref{sec:paramfile}).

For building the codes, first expand the tarball:
\begin{verbatim}
    tar xzf <tarball>.tar.gz
\end{verbatim}
and start the compilation
\begin{verbatim}
    ./bootstrap
    ./configure --with-cpl=<path-to-CPL-library> --prefix=${<basedir>}
       [ LDFLAGS="-L<path-to-lib>" CPPFLAGS="-I<path-to-include>" ]
    make
    make install
\end{verbatim}
After successfull compilation, the executables {\tt molecfit}, {\tt preptable},
{\tt calctrans}, and {\tt corrfilelist} are stored in the \verb|bin/|
directory.

%-------------------------------------------------------------------------------
\subsubsection{Building LNFL/LBLRTM}\label{sec:lbl_installation}
%-------------------------------------------------------------------------------
First, untar the main file:
\begin{quote}
  \begin{small}
    \begin{verbatim}
tar xf lblrtm_v12.1.tar
    \end{verbatim}
  \end{small}
\end{quote}
Then extract the resulting tarballs:
\begin{quote}
  \begin{small}
    \begin{verbatim}
tar zxf aerlnfl_v2.6.tar.gz
tar zxf aerlbl_v12.2.tar.gz
    \end{verbatim}
  \end{small}
\end{quote}
The former contains \ac{LNFL}, which prepares the input. The latter contains
\ac{LBLRTM}, the radiative transfer code.

%-------------------------------------------------------------------------------
\paragraph*{LNFL}
%-------------------------------------------------------------------------------
Change to the directory containing the makefile \verb|make_lnfl|:
\begin{quote}
  \begin{small}
    \begin{verbatim}
cd lnfl/build/
    \end{verbatim}
  \end{small}
\end{quote}
and run
\begin{quote}
  \begin{small}
    \begin{verbatim}
gmake -f make_lnfl linuxGNUsgl
cd ..
    \end{verbatim}
  \end{small}
\end{quote}
which uses the {\tt gfortran} compiler (see
\verb|README.build_instructions|). In the parent directory, you will then
find the final executable \verb|lnfl_v2.6_intel_linux_gnu_sgl|. After the
compilation rename/ relink it to \verb|lnfl|. If the \verb|molecfit| package
is already installed copy it to the <basedir>/bin directory. Note that the
\verb|molecfit| code searches for \verb|lnfl| in this directory (see
Section~\ref{sec:installscript}).

%-------------------------------------------------------------------------------
\paragraph*{LBLRTM}
%-------------------------------------------------------------------------------
Similarly, \ac{LBLRTM} is compiled. Starting from the directory containing the
tarballs, first, move to the makefile location:
\begin{quote}
  \begin{small}
    \begin{verbatim}
cd lblrtm/build/
    \end{verbatim}
  \end{small}
\end{quote}
Then, run
\begin{quote}
  \begin{small}
    \begin{verbatim}
gmake -f make_lblrtm linuxGNUsgl
cd ..
    \end{verbatim}
  \end{small}
\end{quote}
to obtain \verb|lblrtm_v12.1_linux_gnu_sgl| in the parent directory.
Consult \verb|README.build_instructions| in \verb|lblrtm/build/| for more
information on compilation options. It is recommended to use the single
precision as the double precision version caused problems in various tests.
After the compilation rename/relink it to \verb|lblrtm|. If the
\verb|molecfit| package is already installed copy it to the <basedir>/bin
directory. Note that the \verb|molecfit| code searches for \verb|lblrtm| in
this directory (see Section~\ref{sec:installscript}).

%-------------------------------------------------------------------------------
\paragraph*{AER line list}
%-------------------------------------------------------------------------------
The LBLRTM package is delivered with the corresponding line list
{\tt aer\_v\_<version>}, which is the reference line list used by molecfit.
Starting from the directory containing the tarballs, first type
\begin{quote}
  \begin{small}
    \begin{verbatim}
tar zxf aer_v_<version>.tar.gz
    \end{verbatim}
  \end{small}
\end{quote}
to extract the tarball. Finally, copy the line list to the target directory:
\begin{quote}
  \begin{small}
    \begin{verbatim}
cp -v aer_v_<version>/line_file/aer_v_<version> <basedir>/data/hitran/
    \end{verbatim}
  \end{small}
\end{quote}

%-------------------------------------------------------------------------------
\subsection{The automatic way: install script {\tt molecfit\_install.sh}}
\label{sec:installscript}
%-------------------------------------------------------------------------------
We provide an install script, which performs the building and installation
of {\tt molecfit} in five steps:
\begin{itemize}
    \item Step 1: First, a root installation directory {\tt <INST\_DIR>} has to
          be given, within which a subdirectory \mf{} and all deeper
          layers of directories are created.
    \item Step 2: The install directory of the CPL library is requested.
    \item Step 3: Building of \mf{} (see
          {\tt <INST\_DIR>/doc/molecfit\_gcc.log} for the compiler
          output). All required data and configuration files are copied to the
          corresponding target directories {\tt <INST\_DIR>/data} and
          {\tt <INST\_DIR>/config}.
    \item Step 4: Building of the \ac{GRIB} package (see
          {\tt <INST\_DIR>/doc/grib\_gcc.log} for the compiler
          output).
    \item Step 5: Installation of the radiative transfer code
          \ac{LNFL}/\ac{LBLRTM}\footnote{http://rtweb.aer.com/lblrtm\_frame.html}.
          The binaries are stored to {\tt <INST\_DIR>/bin}. In
          addition, a link is created (\verb|lnfl| and \verb|lblrtm|).
\end{itemize}
If the external radiative transfer code is not installed yet, we refer the
reader to Section~\ref{sec:lbl_installation}.

%-------------------------------------------------------------------------------
\subsubsection{Building GRIB}\label{sec:grib_building}
%-------------------------------------------------------------------------------
The \ac{GRIB} package contains all necessary files to automatically download
the \ac{GDAS} profiles from an online archive. In the process, some
supplementary routines to convert Julian to Gregorian dates (and vice versa),
and some perl/shell scripts are required and installed as well. To install the
entire package, simply follow these instructions (assuming the starting point
{\tt <basedir>}):

\begin{verbatim}
tar xvfz grib_package.tar.gz
cd grib_package

cp extract_grib.sh <basedir>/bin
cp get_grib.pl <basedir>/bin
cp get_inv.pl <basedir>/bin

tar xvfz wgrib.tar.gz
cd wgrib
make
cp wgrib  <basedir>/bin
cd ..

tar xvfz wgrib2.tgz
cd grib2
make
cp wgrib2/wgrib2 <basedir>/bin
cd ..

f95 Gregorian_to_JD.f90 -o Gregorian2JD
cp Gregorian2JD  <basedir>/bin

f95 JD_to_Gregorian.f90 -o JD2Gregorian
cp JD2Gregorian  <basedir>/bin
\end{verbatim}
Now the executables {\tt wgrib}, {\tt wgrib2}, {\tt Gregorian2JD},
{\tt JD2Gregorian},  {\tt extract\_grib.sh}, and two perl scripts
{\tt get\_grib.pl}, and {\tt get\_inv.pl} should be located in the
{\tt <basedir>/bin} directory.
