%!TeX root=../tese.tex
%("dica" para o editor de texto: este arquivo é parte de um documento maior)
% para saber mais: https://tex.stackexchange.com/q/78101/183146

% Apague as duas linhas abaixo (elas servem apenas para gerar um
% aviso no arquivo PDF quando não há nenhum dado a imprimir) e
% insira aqui o conteúdo do resumo e abstract do seu trabalho

%!TeX root=../tese.tex
%("dica" para o editor de texto: este arquivo é parte de um documento maior)
% para saber mais: https://tex.stackexchange.com/q/78101/183146

% O resumo é obrigatório, em português e inglês. Este comando também gera
% automaticamente a referência para o próprio documento, conforme as normas
% sugeridas da USP
\begin{resumo}{port}
    Elemento obrigatório, constituído de uma sequência de frases concisas e
    objetivas, em forma de texto.  Deve apresentar os objetivos, métodos empregados,
    resultados e conclusões.  O resumo deve ser redigido em parágrafo único, conter
    no máximo 500 palavras e ser seguido dos termos representativos do conteúdo do
    trabalho (palavras-chave). Deve ser precedido da referência do documento.
    Texto texto texto texto texto texto texto texto texto texto texto texto texto
    texto texto texto texto texto texto texto texto texto texto texto texto texto
    texto texto texto texto texto texto texto texto texto texto texto texto texto
    texto texto texto texto texto texto texto texto texto texto texto texto texto
    texto texto texto texto texto texto texto texto texto texto texto texto texto
    texto texto texto texto texto texto texto texto.
    Texto texto texto texto texto texto texto texto texto texto texto texto texto
    texto texto texto texto texto texto texto texto texto texto texto texto texto
    texto texto texto texto texto texto texto texto texto texto texto texto texto
    texto texto texto texto texto texto texto texto texto texto texto texto texto
    texto texto.
    \end{resumo}
    
    % O resumo é obrigatório, em português e inglês. Este comando também gera
    % automaticamente a referência para o próprio documento, conforme as normas
    % sugeridas da USP
    \begin{resumo}{eng}
    Elemento obrigatório, elaborado com as mesmas características do resumo em
    língua portuguesa. De acordo com o Regimento da Pós-Graduação da USP (Artigo
    99), deve ser redigido em inglês para fins de divulgação. É uma boa ideia usar
    o sítio \url{www.grammarly.com} na preparação de textos em inglês.
    Text text text text text text text text text text text text text text text text
    text text text text text text text text text text text text text text text text
    text text text text text text text text text text text text text text text text
    text text text text text text text text text text text text.
    Text text text text text text text text text text text text text text text text
    text text text text text text text text text text text text text text text text
    text text text.
    \end{resumo}
    
