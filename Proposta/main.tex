\documentclass[11pt,twoside,a4paper]{article}
% ^- openany - open new pages in odd/even page

%packages
\usepackage{url}
\usepackage[T1]{fontenc}
\usepackage[utf8]{inputenc}      % Encoding
\usepackage[portuguese]{babel}   % Correção
%\usepackage{caption}             % Legendas
\usepackage{enumerate}
% Matemática
\usepackage{amsmath}             % Matemática
\usepackage{amsthm, amssymb}     % Matemática

% Gráficos
\usepackage[usenames,dvipsnames]{color}  % Cores
\usepackage[pdftex]{graphicx}   % usamos arquivos pdf/png como figura
\usepackage[usenames,svgnames,dvipsnames,table]{xcolor}

% Desenhos
\usepackage{tikz}
\usepgfmodule{decorations}
\usetikzlibrary{patterns}
\usetikzlibrary{decorations.shapes}
\usetikzlibrary{shapes.geometric}
\usetikzlibrary{decorations.text}
\usetikzlibrary{positioning} % Adjust grid size

% Código-fonte
\usepackage[noend]{algpseudocode}
\usepackage{algorithm}

% Configurações da página
\usepackage{fancyhdr}           % header & footer
\usepackage{float}
\usepackage{setspace}           % espaçamento flexível
\usepackage{indentfirst}        % Identa primeiro parágrafo
\usepackage{makeidx}
\usepackage[nottoc]{tocbibind}  % acrescentamos a  bibliografia/indice/
                                % conteudo no Table of Contents
                                
% Fontes
%\usepackage{helvet}
\renewcommand{\familydefault}{\sfdefault}
\usepackage{type1cm}            % fontes realmente escaláveis
\usepackage{titletoc}
\usepackage{pdflscape}          % Páginas em paisagem
\usepackage{pdfpages}

% Fontes e margens
\usepackage[fixlanguage]{babelbib}
\usepackage[font=small,format=plain,labelfont=bf,up,textfont=it,up]{caption}
\usepackage[a4paper,top=3.0cm,bottom=3.0cm,left=2.0cm,right=2.0cm]{geometry}

% Referências e citações
\usepackage[
    pdftex,
    breaklinks,
    plainpages=false,
    pdfpagelabels,
    pagebackref,
    colorlinks=true,
    citecolor=DarkGreen,
    linkcolor=DarkBlue,
    urlcolor=DarkRed,
    filecolor=green,
    bookmarksopen=true
]{hyperref} 
\usepackage[all]{hypcap} % Soluciona o problema com o hyperref e capitulos
%\usepackage[round,sort,nonamebreak]{natbib} % Citação bibliográfica plainnat-ime
\usepackage{cite}
%\bibpunct{(}{)}{;}{a}{\hspace{-0.7ex},}{,}  % Estilo de citação
%\bibpunct{(}{)}{;}{a}{,}{,}


% Info
%\title{}
\begin{document}
\begin{center}
  \vspace*{3cm}
  
  \Huge
  \textbf{Filtragem de ruído telúrico em sinais astronômicos}

  \vspace{2.5cm}
  \LARGE
  MAC0499 - Trabalho de formatura supervisionado\\
  \vspace{0.3cm}
  \LARGE
  \textit{Proposta de Trabalho}

  
  \vspace{4.3cm}
  \includegraphics[height=4cm,width=3cm]{ime.png}
  \vspace{2cm}
  
  Aluna: \textit{Isabela Blucher}
  
  %\vfill
  
  Orientadores: \textit{Paula Coelho e Marcelo Queiroz}
  
  \vspace{0.8cm}
  
  \Large
  %Instituto de Matemática e Estatística\\
  %Universidade de São Paulo\\
  
\end{center}


\newpage
\tableofcontents
\newpage
\section{Introdução}
\doublespacing

A espectroscopia astronômica é a área da astronomia que tem como objeto de estudo o espectro de radiação eletromagnética proveniente de diversos corpos celestes, como estrelas, planetas, nebulosas, galáxias e núcleos galácticos ativos \cite{wiki:astro_spectroscopy}. 

Estrelas emitem luz em todos os comprimentos de onda, que vão do rádio aos raios-gama. Porém, nem toda luz irradiada pela estrela consegue atingir a Terra. Isso acontece devido à presença de elementos químicos em sua atmosfera, que absorvem a luz emitida e formam linhas de absorção no espectro estelar resultante.

A observação de espectros estelares é relevante devido ao volume de informação que pode ser obtido à partir de estudos espectrais, como composição química, distância, idade, luminosidade e taxa de perda de massa da estrela\cite{wiki:astro_spectroscopy}.

\par A aquisição de espectros estelares se dá por instrumentos denominados espectrógrafos, que dividem a luz irradiada de um objeto celeste em seus comprimentos de onda componentes \cite{spectrograph_aus}. As observações feitas com esse instrumento são capturadas a partir do solo, o que contribui fortemente para a contaminação do sinal, pois ao atravessar a atmosfera terrestre o sinal astronômico interage com gases como vapor de água e oxigênio \cite{seifahrt2010precise}. Esta interação resulta na formação de novas linhas espectrais que se misturam ao sinal original \cite{catanzaro1997high}, resultando em um espectro contaminado pela presença de linhas telúricas. 

\par Para remover as linhas telúricas e recuperar o sinal original, um dos métodos usados é a observação de uma estrela padrão, normalmente uma estrela quente de rotação rápida e com poucas características marcantes além de fortes linhas de hidrogênio \cite{seifahrt2010precise}. A divisão do espectro original pelo espectro da estrela padrão resulta em uma aproximação do espectro da estrela de ciência sem o ruído telúrico \cite{rudolf2016modelling}. Isto é possível devido ao espectro da estrela padrão representar com certa precisão o espectro de transmissão da atmosfera terrestre \cite{ulmer2019telluric}. O grande problema desse método é a falta de eficiência e precisão, pois requer medições suficientemente próximas no tempo e, uma grande quantidade de tempo de uso de telescópio \cite{seifahrt2010precise}. 

%%%%%%%%%%%%%%%%%%%%%%%%%%%%%%%%%%%%%%%%%%%%%%%%%%%%%%%%%%%%%%%%%%%%%%%%%%%%%%%%%%%%%%%%%%%%%%%%%%%%%%%%%%%%%%%%%%%%%%%%%%%%%%%%%%%%%%%%%%%%%%%%%%%%
\section{Objetivo}
\doublespacing
Este trabalho de formatura supervisionado tem como objetivo a implementação  de um \textit{framework} que seja capaz de remover o ruído telúrico de espectros estelares. Para isso é necessário detectar as linhas telúricas, remover o ruído e reconstruir o sinal original, quando possível. 

%%%%%%%%%%%%%%%%%%%%%%%%%%%%%%%%%%%%%%%%%%%%%%%%%%%%%%%%%%%%%%%%%%%%%%%%%%%%%%%%%%%%%%%%%%%%%%%%%%%%%%%%%%%%%%%%%%%%%%%%%%%%%%%%%%%%%%%%%%%%%%%%%%%%
\section{Metodologia}
\doublespacing
Como início do trabalho, está sendo feito um levantamento dos métodos disponíveis em artigos científicos atualmente usados para a resolução do problema. Após o levantamento bibliográfico, será feito um experimento piloto. Este experimento consiste em artificialmente contaminar dados sintéticos e desenvolver um filtro que consiga recuperar ao máximo o sinal original. Esta contaminação será feita usando um software de computação do espectro de transmissão atmosférica \cite{bertaux2014tapas}. Após a contaminação, partindo da suposição de que a atmosfera age como um filtro linear, será feita uma estimativa do seu comportamento, que será aplicada nos sinais contaminados. Para concluir o experimento e quantificar o seu desempenho, será feita a comparação de espectros reais com o resultado do programa.

À partir dos resultados do experimento piloto, serão definidas novas metas e desenvolvimentos futuros para o trabalho. Isso é devido à complexidade do problema, que pode adquirir níveis mais profundos de sofisticação conforme o modelo utilizado para representar a transmissão atmosférica.

A linguagem escolhida para a implementação do experimento piloto foi o \textit{Python}, tanto pela facilidade de escrita quanto pela quantidade de bibliotecas astronômicas e de processamento de sinais existentes.

\newpage
%%%%%%%%%%%%%%%%%%%%%%%%%%%%%%%%%%%%%%%%%%%%%%%%%%%%%%%%%%%%%%%%%%%%%%%%%%%%%%%%%%%%%%%%%%%%%%%%%%%%%%%%%%%%%%%%%%%%%%%%%%%%%%%%%%%%%%%%%%%%%%%%%%%%
\section{Planejamento}
\doublespacing
\subsection{Etapas}
\begin{itemize}
    \item [\textbf{1.}] Estudo da espectroscopia estelar e dos métodos atuais de remoção da contaminação telúrica .
    \item [\textbf{2.}] Obtenção dos dados sintéticos e familiarização com o formato de dados e código para sua leitura e manipulação.
    \item [\textbf{3.}] Contaminação de sinais sintéticos com o software de transmissão atmosférica.
    \item [\textbf{4.}] Estimativa e implementação do filtro linear representativo da atmosfera.
    \item [\textbf{5.}] Testes e resultados do experimento piloto.
    \item [\textbf{6.}] Estudo de extensões/sofisticações do problema e desenvolvimentos futuros.
    \item [\textbf{7.}] Escrita da monografia.
    \item [\textbf{8.}] Preparação do pôster e da apresentação final.
\end{itemize}

\subsection{Cronograma}
\begin{table}[H]
\centering
\caption{}
\label{my-label}
\begin{tabular}{|
>{\columncolor[HTML]{EFEFEF}}l |l|l|l|l|l|l|l|l|}
\hline
\cellcolor[HTML]{9B9B9B}{\color[HTML]{333333} Etapas} & \cellcolor[HTML]{EFEFEF}{\color[HTML]{333333} abr} & \cellcolor[HTML]{EFEFEF}{\color[HTML]{333333} mai} & \cellcolor[HTML]{EFEFEF}{\color[HTML]{333333} jun} & \cellcolor[HTML]{EFEFEF}{\color[HTML]{333333} jul} & \cellcolor[HTML]{EFEFEF}{\color[HTML]{333333} ago} & \cellcolor[HTML]{EFEFEF}{\color[HTML]{333333} set} & \cellcolor[HTML]{EFEFEF}{\color[HTML]{333333} out} & \cellcolor[HTML]{EFEFEF}{\color[HTML]{333333} nov} \\ \hline
{\color[HTML]{000000} 1}                              &                                                 X  &                                                    &                                                    &                                                    &                                                    &                                                    &                                                    &                                                    \\ \hline
{\color[HTML]{000000} 2}                              &                                                  X &                        X                            &                                                    &                                                    &                                                    &                                                    &                                                    &                                                    \\ \hline
{\color[HTML]{000000} 3}                              &                                                   &                      X                              &                                                 X   &                                                    &                                                    &                                                    &                                                    &                                                    \\ \hline
{\color[HTML]{000000} 4}                              &                                                   &                             X                      &                                                   X &               X                                     &                                                    &                                                    &                                                    &                                                    \\ \hline
{\color[HTML]{000000} 5}                              &                                                    &                                                   &                                                X    &              X                                      &                                                    &                                                    &                                                    &                                                    \\ \hline
{\color[HTML]{000000} 6}                              &                                                    &                                                   &                                                   &                        X                            &      X                                              &                                                    &                                                    &                                                    \\ \hline
{\color[HTML]{000000} 7}                              &                                                    &                                                   &                                                  X &                   X                                 &                                                 X   &                   X                                 &                                             X       &                                                    \\ \hline
{\color[HTML]{000000} 8}                              &                                                    &                                                    &                                                   &                                                   &                                                   &                                                   &                                               X     &                                 X                   \\ \hline
\end{tabular}
\end{table}
\newpage
\bibliographystyle{unsrt}
\bibliography{references} % Entries are in the "references.bib" file

%%%%%%%%%%%%%%%%%%%%%%%%%%%%%%%%%%%%%%%%%%%%%%%%%%%%%%%%%%%%%%%%%%%%%%%%

\end{document}