\documentclass[10pt,twoside,a4paper]{article}
% ^- openany - open new pages in odd/even page

%packages
\usepackage[T1]{fontenc}
\usepackage[utf8]{inputenc}      % Encoding
\usepackage[portuguese]{babel}   % Correção
%\usepackage{caption}             % Legendas
\usepackage{enumerate}
% Matemática
\usepackage{amsmath}             % Matemática
\usepackage{amsthm, amssymb}     % Matemática
\newtheorem*{def*}{Definição}
\newtheorem*{invariant}{Invariante}

% Gráficos
\usepackage[usenames,dvipsnames]{color}  % Cores
\usepackage[pdftex]{graphicx}   % usamos arquivos pdf/png como figura
\usepackage[usenames,svgnames,dvipsnames,table]{xcolor}

% Desenhos
\usepackage{tikz}
\usepgfmodule{decorations}
\usetikzlibrary{patterns}
\usetikzlibrary{decorations.shapes}
\usetikzlibrary{shapes.geometric}
\usetikzlibrary{decorations.text}
\usetikzlibrary{positioning} % Adjust grid size

% Código-fonte
\usepackage[noend]{algpseudocode}
\usepackage{algorithm}

% Configurações da página
\usepackage{fancyhdr}           % header & footer
\usepackage{float}
\usepackage{setspace}           % espaçamento flexível
\usepackage{indentfirst}        % Identa primeiro parágrafo
\usepackage{makeidx}
\usepackage[nottoc]{tocbibind}  % acrescentamos a  bibliografia/indice/
                                % conteudo no Table of Contents
                                
% Fontes
%\usepackage{helvet}
\renewcommand{\familydefault}{\sfdefault}
\usepackage{type1cm}            % fontes realmente escaláveis
\usepackage{titletoc}
\usepackage{pdflscape}          % Páginas em paisagem
\usepackage{pdfpages}

% Fontes e margens
\usepackage[fixlanguage]{babelbib}
\usepackage[font=small,format=plain,labelfont=bf,up,textfont=it,up]{caption}
\usepackage[a4paper,top=3.0cm,bottom=3.0cm,left=2.0cm,right=2.0cm]{geometry}

% Referências e citações
\usepackage[
    pdftex,
    breaklinks,
    plainpages=false,
    pdfpagelabels,
    pagebackref,
    colorlinks=true,
    citecolor=DarkGreen,
    linkcolor=DarkBlue,
    urlcolor=DarkRed,
    filecolor=green,
    bookmarksopen=true
]{hyperref} 
\usepackage[all]{hypcap} % Soluciona o problema com o hyperref e capitulos
%\usepackage[round,sort,nonamebreak]{natbib} % Citação bibliográfica plainnat-ime
\usepackage{cite}
%\bibpunct{(}{)}{;}{a}{\hspace{-0.7ex},}{,}  % Estilo de citação
%\bibpunct{(}{)}{;}{a}{,}{,}


% Info
%\title{}
\begin{document}
\begin{center}
  \vspace*{3cm}
  
  \Huge
  \textbf{Filtragem de ruído telúrico em sinais astronômicos}

  \vspace{2.5cm}
  \LARGE
  MAC0499 - Trabalho de formatura supervisionado\\
  \vspace{0.3cm}
  \LARGE
  \textit{Proposta de Trabalho}

  
  \vspace{4.3cm}
  \includegraphics[height=4cm,width=3cm]{ime.png}
  \vspace{2cm}
  
  Aluna: \textit{Isabela Blucher}
  
  %\vfill
  
  Orientadores: \textit{Paula Coelho e Marcelo Queiroz}
  
  \vspace{0.8cm}
  
  \Large
  %Instituto de Matemática e Estatística\\
  %Universidade de São Paulo\\
  
\end{center}


\newpage
\tableofcontents
\newpage
\section{Introdução}
\doublespacing


% Proveniente da área de análise de algoritmos, a geometria computacional é a área da computação que pode ser definida como o estudo sistemático de algoritmos e estruturas de dados para objetos geométricos, com foco em algoritmos exatos que são assintóticamente rápidos \cite{dbvkos}. A geometria computacional tem aplicações em diversas áreas como: computação gráfica, reconhecimento de padrões, processamento de imagens, robótica, metalurgia, manufatura e estatística \cite{cormen}. Tais problemas são tratados com o uso de objetos geométricos primitivos como: pontos, retas, segmentos de reta, polígonos.
% \par Este trabalho será focado num problema específico da área de geometria computacional classificado como um problema de busca geométrico. Nesses tipos de problemas a entrada é formada por duas partes: o espaço de busca e a consulta que queremos realizar. É comum que seja necessário realizar um pré-processamento no espaço de busca para que possamos responder várias consultas seguidas de forma eficiente. Alguns exemplos de problemas desse tipo são: localização de pontos, localização de vizinhos mais próximos, \textit{ray tracing} e busca em intervalos.
%%%%%%%%%%%%%%%%%%%%%%%%%%%%%%%%%%%%%%%%%%%%%%%%%%%%%%%%%%%%%%%%%%%%%%%%%%%%%%%%%%%%%%%%%%%%%%%%%%%%%%%%%%%%%%%%%%%%%%%%%%%%%%%%%%%%%%%%%%%%%%%%%%%%
\section{Objetivo}
\doublespacing


% \par Neste trabalho de formatura supervisionado será feita uma implementação para resolver o seguinte problema: dado um conjunto de segmentos no plano, identificar rapidamente todos os segmentos (ou pontos) contidos numa janela retangular de lados paralelos aos eixos. \par O objetivo será escrever uma biblioteca com implementações dos algoritmos e estruturas de dados relacionados a esse problema assim como a análise rigorosa de tempo e espaço dos algoritmos implementados.
%%%%%%%%%%%%%%%%%%%%%%%%%%%%%%%%%%%%%%%%%%%%%%%%%%%%%%%%%%%%%%%%%%%%%%%%%%%%%%%%%%%%%%%%%%%%%%%%%%%%%%%%%%%%%%%%%%%%%%%%%%%%%%%%%%%%%%%%%%%%%%%%%%%%
\section{Metodologia}
\doublespacing

% \par Já foi iniciado o estudo do algoritmo implementado em \cite{alvaro}, assim como da literatura associada. A linguagem escolhida para a implementação foi \textit{python}, tanto pela facilidade de escrita quanto de se  mostrar graficamente os resultados obtidos. Será feita uma biblioteca com os algoritmos e estruturas de dados utilizadas que será disponibilizada no \href{http://github.com/mlordx/mac0499}{gitHub}.
\newpage
%%%%%%%%%%%%%%%%%%%%%%%%%%%%%%%%%%%%%%%%%%%%%%%%%%%%%%%%%%%%%%%%%%%%%%%%%%%%%%%%%%%%%%%%%%%%%%%%%%%%%%%%%%%%%%%%%%%%%%%%%%%%%%%%%%%%%%%%%%%%%%%%%%%%
\section{Planejamento}
\doublespacing
\subsection{Etapas}
% \begin{itemize}
% \item[\textbf{1.}] Implementação das primitivas geométricas.
% \item[\textbf{2.}] Estudar o problema no caso unidimensional.
% \item[\textbf{3.}] Implementar o algoritmo para o caso unidimensional com diferentes estruturas de dados e fazer comparações de perfomance e espaço.
% \item[\textbf{4.}] Estudar o problema no caso bidimensional.
% \item[\textbf{5.}] Implementar o algoritmo para o caso bidimensional com diferentes estruturas de dados e fazer comparações de perfomance e espaço.
% \item[\textbf{6.}] Estudar possíveis extensões do problema (pontos em movimento, n-dimensões, etc).
% \item[\textbf{7.}] Implementar tais extensões e fazer comparações de perfomance e espaço das estruturas de dados utilizadas.
% \item[\textbf{8.}] Escrever a monografia.
% \item[\textbf{9.}] Preparar pôster e apresentação.
% \end{itemize}
\subsection{Cronograma}
\begin{table}[H]
\centering
\caption{}
\label{my-label}
\begin{tabular}{|
>{\columncolor[HTML]{EFEFEF}}l |l|l|l|l|l|l|l|l|}
\hline
\cellcolor[HTML]{9B9B9B}{\color[HTML]{333333} Etapas} & \cellcolor[HTML]{EFEFEF}{\color[HTML]{333333} abr} & \cellcolor[HTML]{EFEFEF}{\color[HTML]{333333} mai} & \cellcolor[HTML]{EFEFEF}{\color[HTML]{333333} jun} & \cellcolor[HTML]{EFEFEF}{\color[HTML]{333333} jul} & \cellcolor[HTML]{EFEFEF}{\color[HTML]{333333} ago} & \cellcolor[HTML]{EFEFEF}{\color[HTML]{333333} set} & \cellcolor[HTML]{EFEFEF}{\color[HTML]{333333} out} & \cellcolor[HTML]{EFEFEF}{\color[HTML]{333333} nov} \\ \hline
{\color[HTML]{000000} 1}                              &                                                   &                                                    &                                                    &                                                    &                                                    &                                                    &                                                    &                                                    \\ \hline
{\color[HTML]{000000} 2}                              &                                                   &                                                    &                                                    &                                                    &                                                    &                                                    &                                                    &                                                    \\ \hline
{\color[HTML]{000000} 3}                              &                                                   &                                                    &                                                    &                                                    &                                                    &                                                    &                                                    &                                                    \\ \hline
{\color[HTML]{000000} 4}                              &                                                   &                                                   &                                                    &                                                    &                                                    &                                                    &                                                    &                                                    \\ \hline
{\color[HTML]{000000} 5}                              &                                                    &                                                   &                                                    &                                                    &                                                    &                                                    &                                                    &                                                    \\ \hline
{\color[HTML]{000000} 6}                              &                                                    &                                                   &                                                   &                                                    &                                                    &                                                    &                                                    &                                                    \\ \hline
{\color[HTML]{000000} 7}                              &                                                    &                                                   &                                                   &                                                    &                                                    &                                                    &                                                    &                                                    \\ \hline
{\color[HTML]{000000} 8}                              &                                                    &                                                    &                                                   &                                                   &                                                   &                                                   &                                                    &                                                    \\ \hline
{\color[HTML]{000000} 9}                              &                                                    &                                                    &                                                   &                                                   &                                                   &                                                   &                                                   &                                                   \\ \hline
\end{tabular}
\end{table}
\newpage
\bibliography{bla}{}
\bibliographystyle{plain}

%%%%%%%%%%%%%%%%%%%%%%%%%%%%%%%%%%%%%%%%%%%%%%%%%%%%%%%%%%%%%%%%%%%%%%%%

\end{document}