%% ------------------------------------------------------------------------- %%
\chapter{Conclusão}
\label{cap:conclusoes}

\section{Fechamento}

% Breve resumo do tema
Este trabalho possibilitou explorar um problema que une as áreas da Astronomia e da Computação, ao estudar observações de espectros estelares desde uma perspectiva computacional. O problema da contaminação telúrica em espectros estelares é uma fonte de distorções em observações astronômicas feitas do solo, e continua sendo um impedimento para o avanço da pesquisa com estes dados. Linhas de absorção de diferentes perfis e origens moleculares podem ser difíceis de remover com precisão, porém, essa remoção é fundamental para interpretar os espectros capturados. 

% o que foi analisado no desenvolvimento da monografia
A investigação exploratória dos dados aqui apresentada é preliminar e objetivou identificar problemas e possível soluções com o intuito de minimizar a interferência atmosférica nestas observações. O desenvolvimento desta monografia tornou possível a exploração de dados reais de estrelas e a tentativa de propor uma solução para um dos problemas conhecidos destas observações, os desalinhamentos encontrados entre os espectros estelares e telúricos.

% Demonstrar se os objetivos propostos na seção de introdução da monografia foram concluídos. Se as perguntas e problemas apresentados inicialmente foram respondidas e/ou esclarecidas
A abordagem escolhida com o algoritmo do \textit{Dynamic Time Warping} tinha como objetivo explorar as similaridades entre os sinais estelar e telúrico, especificamente a presença de linhas de absorção em comprimentos de onda específicos, para reconstruir um espectro telúrico realinhado, de modo a minimizar a perda de informação e os artefatos criados na divisão simples dos espectros. Embora as linhas de absorção possam fornecer pistas visuais importantes para os olhos de um pesquisador treinado, as diferenças importantes entre os espectros estelar e telúrico revelou as limitações daquele algoritmo de alinhamento deste tipo de sequências visto que o DTW distorce fortemente o espectro telúrico para minimizar a distância à referência estelar observada, o que não corresponde a uma solução ideal e generalizável para as observações espectrais e seus respectivos referenciais telúricos.

% motivação para os trabalhos futuros
Apesar das dificuldades encontradas, o processo pretendido de familiarização com os dados dos espectros estelares, com algumas das interpretações associadas às configurações neles encontradas, bem como com as técnicas de manipulação desse tipo de informação foi muito proveitoso, permitindo a identificação de uma série de questões para pesquisa que serão exploradas em trabalhos futuros.

% Importancia da pesquisa interdisciplinar para a comunidade científica
Esta pesquisa tem interesse, entre outras razões, pela grande quantidade de problemas observacionais e instrumentais conhecidos -- além dos desalinhamentos nos sinais -- que impactam a qualidade dos dados disponíveis para pesquisas astronômicas. Além disso, esta pesquisa tem importância pela sua natureza interdisciplinar, pelo potencial explorável na aplicação de teorias e métodos computacionais na solução de problemas de interesse da comunidade astronômica, e por ajudar a criar pontes de comunicação através de um vocabulário compartilhado entre essas duas importantes áreas do conhecimento.

\section{Trabalhos futuros}

Durante o desenvolvimento deste trabalho interdisciplinar, uma série de ideias de pesquisa surgiram, passíveis de serem exploradas no futuro. A lista abaixo pretende servir como memória para possíveis extensões da pesquisa aqui iniciada.


\begin{itemize}


%\item \pc{nao ficou claro para mim se o DTW falhou mesmo, para a ideia original, ou se é conveniente explorar limitando o caminho (por exemplo a sempre dar passos na diagonal}.

\item O DTW, pela sua característica de maximizar a similaridade entre dois sinais através de realinhamentos, poderia ter utilidade em uma etapa da redução de dados, chamada de calibração em comprimento de onda. Nessa etapa procura-se encontrar uma função -- possivelmente não-linear -- que converte os índices do espectro na escala física dos comprimentos de onda. 

\item Na modelagem do processo de aquisição através de espectrografia astronômica, seria plausível tentar modelar a assinatura instrumental como um filtro, e buscar estimar parâmetros desse filtro através de pares de espectros observados e corrigidos, ou a partir de observações obtidas por instrumentos diferentes em condições similares (posição, direção, condições climáticas).%O DTW trata apenas de diferenças de alinhamento. No entanto podemos pensar na assinatura instrumental de um modo mais genérico \pc{... foi aquela conversa em termos de (cópia dos logs da isa):  "A imagem acima mostra o trecho anterior da lousa a uma parte a mais que representa o raciocínio de inserir a assinatura instrumental do CCD na operação entre o sinal estelar e a atmosfera. ", "Outro caminho seria estimar Mh e Mh-1..." (copiado da segunda foto da lousa do log de 27/06/1). }

\item Pode-se buscar uma descrição \textit{data-driven} do espectro telúrico, em contrapartida à modelagem física de transferência radiativa. Nessa abordagem, pode-se partir de uma grande quantidade de espectros telúricos disponíveis em \textit{archives} de astronomia, para generalizar um template telúrico em termos de filtros gaussianos. As variações de intensidade do sinal telúrico (ligadas às condições atmosféricas) e de resolução espectral (ligadas ao instrumentos) poderiam ser parametrizadas em termos de funções agnósticas às particularidades físicas do problema.

\item \pc{copy/paste do comentario do rafael, se voces acharem que faz sentido explorar: "After some discussion with Paula, I understood this project as a blind source separation, in which the source spectrum is convoluted with the telluric lines, so we should be able to disentangle using some sort of ICA and extensions. This can be quite feasible in theory and are useful for virtually any ground based survey."}

\item \pc{Do log de 27/07: "o professor Marcelo pensou na profundidade dos sulcos dos sinais e como não estávamos olhando para essa dimensão ainda. Ele chegou à conclusão que poderíamos achar uma constante ɑ > 0 que consegue representar os sulcos e modificar a escala dos sinais para que a divisão do observado pelo telúrico gere resultados melhores. Essa diferença nos sulcos dos espectros pode tanto ser causada por mudanças nas condições observacionais quanto por mudanças no processo de redução de dados dos sinais, o que implica que achar esse ɑ pode ser generalizável para vários casos. O problema de achar esse valor é um problema de maximização."}

\item \pc{O ultimo log registrado diz "Decidimos continuar com os dados do X-Shooter para testes, e com a região dos 8000A (fim do visível, começo do infravermelho: regiões de mais interesse em termos de contaminação telúrica). Vamos aplicar três técnicas de pré-processamento nos dados:
Whitening
Filtro passa-alta
Filtro da mediana
Aplicar esses filtros e usar essas versões na DTW e também entender o comportamento dessas técnicas por si mesmas. 
Ainda vamos usar os espectros recorrigidos como ground-truth e vamos usar como métrica de avaliação a diferença relativa $(norma(dif)^2 / norma(f)^2)$." Nao me lembro das discussoes se iriamos retomar essa questao dos filtros ou nao.}

\end{itemize}


% \section{Agradecimentos}