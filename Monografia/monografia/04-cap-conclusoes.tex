%% ------------------------------------------------------------------------- %%
\chapter{Conclusão}
\label{cap:conclusoes}

\section{Fechamento}

% Breve resumo do tema e o que foi analisado no desenvolvimento da monografia
Este trabalho possibilitou explorar um problema que une as áreas da Astronomia e da Computação, ao estudar dados estelares de uma perspectiva computacional. O problema da contaminação telúrica em espectros astronômicos é uma fonte de distorções em observações astronômicas feitas do solo, e continua sendo um impedimento para atingirmos a qualidade ideal destes dados em pesquisas astronômicas. O desenvolvimento desta monografia tornou possível a exploração de dados reais de estrelas e a tentativa de propor uma solução para um dos problemas conhecidos destas observações, os desalinhamentos encontrados entre os espectros estelares e telúricos.  

% Explicar a importância do tema, qual sua relevância para o meio acadêmico
O tema da contaminação telúrica é relevante pois até hoje em dia representa um problema para o observador. Linhas de absorção de diferentes perfis e origens moleculares podem ser difíceis de remover com precisão, porém, essa remoção é fundamental para interpretar espectros, não só estelares, mas de outros objetos celestes. Inicialmente foi feita uma investigação exploratória dos dados para entender problemas e possível soluções com o intuito de minimizar a interferência atmosférica nestas observações.

% Demonstrar se os objetivos propostos na seção de introdução da monografia foram concluídos. Se as perguntas e problemas apresentados inicialmente foram respondidas e/ou esclarecidas
A abordagem escolhida com o algoritmo do \textit{Dynamic Time Warping} tinha como objetivo encontrar similaridades entre os sinais estelar e telúrico para reconstruir um espectro telúrico realinhado, de modo a minimizar a quantidade de informação perdida e artefatos criados na divisão simples dos espectros. Embora a ideia de explorar uma métrica da similaridade entre os sinais possa a priori explorar os desalinhamentos nos espectros, o algoritmo selecionado é melhor aplicado em sinais que representam informações parecidas e que possuem formatos similares. Isso resultou que quando aplicamos a técnica entre um espectro estelar e um telúrico, o DTW distorceu o espectro de ciência para minimizar a distância à referência telúrica, o que não apresenta uma solução ideal e generalizável para as observações espectrais e seus respectivos referenciais telúricos.   

% Importancia da pesquisa interdisciplinar para a comunidade científica
Esta pesquisa tem potencial de futura exploração e evolução, dada a quantidade de problemas observacionais e instrumentais -- além dos desalinhamentos nos sinais -- que impactam a qualidade do espectro observado para pesquisas astronômicas. Mas além disso, esta pesquisa tem importância pela sua natureza interdisciplinar, pelo esforço de aplicar teoria e métodos computacionais no auxílio da comunidade astronômica e por criar um vocabulário unificado entre importantes áreas do conhecimento.

\section{Trabalhos futuros}

Durante o desenvolvimento deste trabalho interdisciplinar, uma séria de idéias surgiram, passíveis de serem exploradas no futuro. 


\begin{itemize}


\item \pc{nao ficou claro para mim se o DTW falhou mesmo, para a ideia original, ou se é conveniente explorar limitando o caminho (por exemplo a sempre dar passos na diagonal}.

\item O DTW, pela sua característica de maximizar a identidade entre dois sinais, pode ter uma aplicação em uma etapa da redução de dados, chamada de calibração em comprimento de onda. Nessa etapa procura-se encontrar a função que converte a escala física dos pixeis em comprimentos de onda. 

\item O DTW trata apenas de diferenças de alinhamento. No entanto podemos pensar na assinatura instrumental de um modo mais genérico \pc{... foi aquela conversa em termos de (cópia dos logs da isa):  "A imagem acima mostra o trecho anterior da lousa a uma parte a mais que representa o raciocínio de inserir a assinatura instrumental do CCD na operação entre o sinal estelar e a atmosfera. ", "Outro caminho seria estimar Mh e Mh-1..." (copiado da segunda foto da lousa do log de 27/06/1). }

\item Pode-se buscar uma descrição \textit{data-driven} do espectro telúrico, em contrapartida à modelagem física de transferência radiativa. Nessa abordagem, pode-se partir de uma grande quantidade de espectros telúricos disponíveis em \textit{archives} de astronomia, para generalizar um template telúrico em termos de filtros gaussianos. As variações de intensidade do sinal telúrico (ligadas às condições atmosféricas) e de resolução espectral (ligadas ao instrumentos) podem ser parametrizadas em termos de funções agnósticas às particularidades físicas do problema.

\item \pc{copy/paste do comentario do rafael, se voces acharem que faz sentido explorar: "After some discussion with Paula, I understood this project as a blind source separation, in which the source spectrum is convoluted with the telluric lines, so we should be able to disentangle using some sort of ICA and extensions. This can be quite feasible in theory and are useful for virtually any ground based survey."}

\item \pc{Do log de 27/07: "o professor Marcelo pensou na profundidade dos sulcos dos sinais e como não estávamos olhando para essa dimensão ainda. Ele chegou à conclusão que poderíamos achar uma constante ɑ > 0 que consegue representar os sulcos e modificar a escala dos sinais para que a divisão do observado pelo telúrico gere resultados melhores. Essa diferença nos sulcos dos espectros pode tanto ser causada por mudanças nas condições observacionais quanto por mudanças no processo de redução de dados dos sinais, o que implica que achar esse ɑ pode ser generalizável para vários casos. O problema de achar esse valor é um problema de maximização."}

\item \pc{O ultimo log registrado diz "Decidimos continuar com os dados do X-Shooter para testes, e com a região dos 8000A (fim do visível, começo do infravermelho: regiões de mais interesse em termos de contaminação telúrica). Vamos aplicar três técnicas de pré-processamento nos dados:
Whitening
Filtro passa-alta
Filtro da mediana
Aplicar esses filtros e usar essas versões na DTW e também entender o comportamento dessas técnicas por si mesmas. 
Ainda vamos usar os espectros recorrigidos como ground-truth e vamos usar como métrica de avaliação a diferença relativa $(norma(dif)^2 / norma(f)^2)$." Nao me lembro das discussoes se iriamos retomar essa questao dos filtros ou nao.}

\end{itemize}


% \section{Agradecimentos}