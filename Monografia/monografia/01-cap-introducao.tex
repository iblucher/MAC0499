%% ------------------------------------------------------------------------- %%
\chapter{Introdução}
\label{cap:introducao}

% computação como ferramenta importante para outas áreas da ciência 
A Computação é a área do conhecimento que estuda técnicas, metodologias e instrumentos computacionais com o objetivo de automatizar processos e desenvolver soluções para uma variedade de problemas. Um dos grandes usos da Computação atualmente é como ferramenta no avanço de pesquisas científicas em diversas áreas, por exemplo em ciências naturais, nas áreas de física, biologia, geografia e etc.

% computação na astronomia
A ênfase deste trabalho é na aplicação de métodos computacionais em Astronomia. Com o passar do tempo e com uma constante evolução da tecnologia, as ferramentas astronômicas mudaram, desde como o universo e os objetos celestes são observados até como estes dados observacionais são processados e armazenados. Atualmente existem diversos desafios em astronomia que envolvem áreas da computação como aprendizado de máquina, ciência de dados, processamento de sinais digitais e \textit{Big Data}. 

% o que este trabalho se propõe em fazer (recorte especifico da aplicacao da comp em astro)
O recorte específico da computação na astronomia tratado nesta monografia é referente a espectroscopia estelar e remoção de contaminação telúrica nestes dados, apresentando métodos usados atualmente para fazer a correção telúrica dos sinais e experimentos que testam métodos de processamento de sinais digitais de forma a melhorar a qualidade desta correção.

\section{Contextualização}

% contexto das duas grandes áreas da astronomia
A astronomia pode ser divida em duas grandes áreas de estudo: a astronomia teórica e a astronomia observacional. Em contraste com a astronomia teórica, que tem como foco a modelagem de fenômenos astronômicos e o cálculo de suas implicações no universo, a astronomia observacional tem como objetivo a observação e o registro de dados sobre o universo observável e o estudo e interpretação destas observações.  

% astronomia observacional e uso de instrumentos
Dentro do cenário da astronomia observacional, são estudados uma grande variedade objetos celestes através de suas fontes de informação: a emissão de luz, ou, radiação eletromagnética. Para observar e capturar esta radiação, astrônomos usam uma variedade de instrumentos que possuem a capacidade de transformar um sinal luminoso em informação cientificamente interpretável.

% espectros: como são observados e usados em pesquisa
Um exemplo de instrumento observacional muito usado em pesquisas astronômicas é o espectrógrafo, que transforma uma observação celeste em um espectro, através da divisão da radiação eletromagnética emitida em seus comprimentos de onda. O espectro de um objeto celeste, e no caso deste trabalho, das estrelas, é uma informação muito relevante em pesquisa, pois está relacionado diretamente a diversas características sobre a estrela, como sua composição química, temperatura e luminosidade.

% dificuldades nas observações
No entanto, a observação de espectros estelares não é um processo perfeito e o sinal capturado pode sofrer distorções tanto instrumentais quanto físicas. A radiação eletromagnética emitida por uma estrela, quando entra em contato com a atmosfera terrestre interage com as moléculas presentes. Esta interação cria artefatos no espectro capturado da estrela na forma de novas linhas de absorção, o que resulta em um espectro contaminado pela atmosfera terrestre.

% o que o trabalho quer fazer
Esta interação da atmosfera com a emissão estelar é chamada de contaminação telúrica e é um fenômeno que interfere com todas as observações estelares feitas a partir do solo terrestre. O objetivo geral deste trabalho consiste em estudar métodos existentes usados por astrônomos para corrigir espectros estelares de contaminação telúrica e investigar métodos computacionais com o potencial de auxiliar o este processo de correção .


\section{Estrutura do Trabalho}

% como estão separados os capítulos
O capítulo~\ref{cap:fundamentacao-teorica} apresenta a base teórica astronômica por trás da espectroscopia estelar e seu sistema de classificação, o processo de captura e redução de dados e a ciência da contaminação telúrica de espectros estelares. Além disso, é apresentado a base computacional da definição do problema de similaridade e alinhamento de sinais digitais e algoritmos usados neste campo.

O capítulo~\ref{cap:desenvolvimentos} apresenta especificações sobre o formato de dados com o qual foi trabalhado, os conjuntos de dados públicos ou fornecidos por comunicação particular e as ferramentas computacionais utilizadas para a leitura e manipulação destes dados. Além disso, são descritos os experimento realizados no desenvolvimento deste trabalho e seus resultados.

O capítulo~\ref{cap:conclusoes} expõe as conclusões finais sobre a pesquisa realizada, menciona as dificuldades encontradas e confronta os resultados com os objetivos iniciais.
% perspectivas devia ser um cap próprio ou pode ser uma seção da conclusão?
