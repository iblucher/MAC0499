%% ------------------------------------------------------------------------- %%
\chapter{Fundamentação Teórica}
\label{cap:fundamentacao-teorica}

O problema de remoção de ruído telúrico em sinais astronômicas é de natureza interdisciplinar.
Neste capítulo é descrita a teoria da astronomia que caracteriza o problema da contaminação telúrica de sinais astronômicos. Também é descrita a teoria de sinais digitais de técnicas computacionis aplicadas nos experimentos.

\section{Espectroscopia Astronômica}

A espectroscopia astronômica é a área da astronomia que tem como objeto de estudo o espectro de radiação eletromagnética proveniente de diversos corpos celestes, como estrelas, planetas, nebulosas, galáxias e núcleos galácticos ativos. 

\section{Processamento de Sinais}
\subsection{Whitening}
\subsection{Filtro Passa-Alta}
\subsection{Filtro da Mediana}
\subsection{DTW}

\section{Contaminação Telúrica}
\subsection{Métodos atuais}
% Uma monografia deve ter um capítulo inicial que é a Introdução e um
% capítulo final que é a Conclusão. Entre esses dois capítulos poderá
% ter uma sequência de capítulos que descrevem o trabalho em detalhes.
% Após o capítulo de conclusão, poderá ter apêndices e ao final deverá
% ter as referências bibliográficas.


% Para a escrita de textos em Ciência da Computação, o livro de Justin Zobel, 
% \emph{Writing for Computer Science} \citep{zobel:04} é uma leitura obrigatória. 
% O livro \emph{Metodologia de Pesquisa para Ciência da Computação} de 
% \citet{waz:09} também merece uma boa lida.

% O uso desnecessário de termos em lingua estrangeira deve ser evitado. No entanto,
% quando isso for necessário, os termos devem aparecer \emph{em itálico}.

% \begin{small}
% \begin{verbatim}
% Modos de citação:
% indesejável: [AF83] introduziu o algoritmo ótimo.
% indesejável: (Andrew e Foster, 1983) introduziram o algoritmo ótimo.
% certo : Andrew e Foster introduziram o algoritmo ótimo [AF83].
% certo : Andrew e Foster introduziram o algoritmo ótimo (Andrew e Foster, 1983).
% certo : Andrew e Foster (1983) introduziram o algoritmo ótimo.
% \end{verbatim}
% \end{small}

% Uma prática recomendável na escrita de textos é descrever as legendas das
% figuras e tabelas em forma auto-contida: as legendas devem ser razoavelmente
% completas, de modo que o leitor possa entender a figura sem ler o texto onde a
% figura ou tabela é citada.  

% Apresentar os resultados de forma simples, clara e completa é uma tarefa que
% requer inspiração. Nesse sentido, o livro de \citet{tufte01:visualDisplay},
% \emph{The Visual Display of Quantitative Information}, serve de ajuda na
% criação de figuras que permitam entender e interpretar dados/resultados de forma
% eficiente.

