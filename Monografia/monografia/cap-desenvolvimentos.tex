%% ------------------------------------------------------------------------- %%
\chapter{Metodologia}
\label{cap:desenvolvimentos}

Este trabalho tem o objetivo de desenvolver uma solução que seja capaz de remover a contaminação telúrica de espectros astronômicos capturados a partir do solo. Para fazer isso é necessário um conhecimento extenso sobre os dados coletados, desde como foram criados e compostos até sua semântica e nuances astronômicas.  

Neste capítulo são descritos os experimentos realizados durante o desenvolvimento do trabalho que foram úteis para compreender e visualizar os espectros de ciência e telúrico, testar o realinhamento de sinais e aprender características fundamentais dos espectros na solução do problema da contaminação telúrica.  

\section{Formato de dados FITS}

O formato de dados utilizado nos experimentos é o FITS ou \textit{Flexible Image Transfer System}. Este formato de arquivo digital facilita o armazenamento, processamento e transmissão de dados científicos e foi projetado para armazenar conjuntos de dados $n$-dimensionais que consistem em matrizes e tabelas. Este é o formato de dados mais utilizado na astronomia, e inclusive possui recursos adicionais para incluir informações de calibração fotométrica e espacial e metadados de origem do arquivo \citep{pence2010definition}.

Um arquivo FITS é composto por segmentos chamados de HDU (\textit{Header-Data Units}). Um arquivo FITS deve necessariamente ter um \textit{Primary HDU}, que armazena os dados científicos principais e pode ser acompanhado de HDUs adicionais, categorizados entre extensões de imagens, extensões de tabelas ASCII e extensões de tabelas binárias \citep{nasa-fits}.

\section{Divisão do sinal estelar pelo telúrico}

\section{DTW no sinal estelar}

\section{DTW no sinal atmosférico}