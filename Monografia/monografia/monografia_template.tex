% Arquivo LaTeX de exemplo de monografia para a disciplina MAC0499
% 
% Adaptado em julho/2015 a partir do
%
% ---------------------------------------------------------------------------- %
% Arquivo LaTeX de exemplo de dissertação/tese a ser apresentados à CPG do IME-USP
%
% Versão 5: Sex Mar  9 18:05:40 BRT 2012
%
% Criação: Jesús P. Mena-Chalco
% Revisão: Fabio Kon e Paulo Feofiloff


\documentclass[12pt,twoside,a4paper]{book}


% ---------------------------------------------------------------------------- %
% Pacotes 
\usepackage[T1]{fontenc}
\usepackage[brazil]{babel}
\usepackage[utf8]{inputenc}
\usepackage[pdftex]{graphicx}           % usamos arquivos pdf/png como figuras
\usepackage{setspace}                   % espaçamento flexível
\usepackage{indentfirst}                % indentação do primeiro parágrafo
\usepackage{makeidx}                    % índice remissivo
\usepackage[nottoc]{tocbibind}          % acrescentamos a bibliografia/indice/conteudo no Table of Contents
% \usepackage{courier} % usa o Adobe Courier no lugar de Computer Modern Typewriter
\usepackage{lmodern}
\usepackage{type1cm}                    % fontes realmente escaláveis
\usepackage{listings}                   % para formatar código-fonte (ex. em Java)
\usepackage{titletoc}
%\usepackage[bf,small,compact]{titlesec} % cabeçalhos dos títulos: menores e compactos
\usepackage[fixlanguage]{babelbib}
\usepackage[font=small,format=plain,labelfont=bf,up,textfont=it,up]{caption}
\usepackage[usenames,svgnames,dvipsnames]{xcolor}
\usepackage[a4paper,top=2.54cm,bottom=2.0cm,left=2.0cm,right=2.54cm]{geometry} % margens
%\usepackage[pdftex,plainpages=false,pdfpagelabels,pagebackref,colorlinks=true,citecolor=black,linkcolor=black,urlcolor=black,filecolor=black,bookmarksopen=true]{hyperref} % links em preto
\usepackage[pdftex,plainpages=false,pdfpagelabels,pagebackref,colorlinks=true,citecolor=DarkGreen,linkcolor=NavyBlue,urlcolor=DarkRed,filecolor=green,bookmarksopen=true]{hyperref} % links coloridos
\usepackage[all]{hypcap}                    % soluciona o problema com o hyperref e capitulos
\usepackage[round,sort,nonamebreak]{natbib} % citação bibliográfica textual(plainnat-ime.bst)
\usepackage{emptypage}  % para não colocar número de página em página vazia
\fontsize{60}{62}\usefont{OT1}{cmr}{m}{n}{\selectfont}
\usepackage{amsmath}
\usepackage{algorithm}
\usepackage{algpseudocode}
\usepackage{graphicx}
\usepackage{subfig}
\usepackage{subcaption}


% ---------------------------------------------------------------------------- %
% Cabeçalhos similares ao TAOCP de Donald E. Knuth
\usepackage{fancyhdr}
\pagestyle{fancy}
\fancyhf{}
\renewcommand{\chaptermark}[1]{\markboth{\MakeUppercase{#1}}{}}
\renewcommand{\sectionmark}[1]{\markright{\MakeUppercase{#1}}{}}
\renewcommand{\headrulewidth}{0pt}

% ---------------------------------------------------------------------------- %
\graphicspath{{./figuras/}}             % caminho das figuras (recomendável)
\frenchspacing                          % arruma o espaço: id est (i.e.) e exempli gratia (e.g.) 
\urlstyle{same}                         % URL com o mesmo estilo do texto e não mono-spaced
\makeindex                              % para o índice remissivo
\raggedbottom                           % para não permitir espaços extra no texto
\fontsize{60}{62}\usefont{OT1}{cmr}{m}{n}{\selectfont}
\cleardoublepage
\normalsize

% ---------------------------------------------------------------------------- %
% Opções de listing usados para o código fonte
% Ref: http://en.wikibooks.org/wiki/LaTeX/Packages/Listings
\lstset{ %
language=Java,                  % choose the language of the code
basicstyle=\footnotesize,       % the size of the fonts that are used for the code
numbers=left,                   % where to put the line-numbers
numberstyle=\footnotesize,      % the size of the fonts that are used for the line-numbers
stepnumber=1,                   % the step between two line-numbers. If it's 1 each line will be numbered
numbersep=5pt,                  % how far the line-numbers are from the code
showspaces=false,               % show spaces adding particular underscores
showstringspaces=false,         % underline spaces within strings
showtabs=false,                 % show tabs within strings adding particular underscores
frame=single,	                % adds a frame around the code
framerule=0.6pt,
tabsize=2,	                    % sets default tabsize to 2 spaces
captionpos=b,                   % sets the caption-position to bottom
breaklines=true,                % sets automatic line breaking
breakatwhitespace=false,        % sets if automatic breaks should only happen at whitespace
escapeinside={\%*}{*)},         % if you want to add a comment within your code
backgroundcolor=\color[rgb]{1.0,1.0,1.0}, % choose the background color.
rulecolor=\color[rgb]{0.8,0.8,0.8},
extendedchars=true,
xleftmargin=10pt,
xrightmargin=10pt,
framexleftmargin=10pt,
framexrightmargin=10pt
}

% ---------------------------------------------------------------------------- %
% Definicoes dos autores

\newcommand{\pc}[1]{\texttt{\textcolor{magenta}{[#1]}}}
\newcommand{\revise}[1]{{\textcolor{orange}{\bf #1}}}
\newcommand{\mqz}[1]{\\\medskip\noindent\begin{minipage}{\textwidth}\texttt{\textcolor{green}{[\qquad #1]}}\end{minipage}\\\medskip}
\newcommand{\mqzinline}[1]{\texttt{\textcolor{green}{[#1]}}}

\newcommand{\attrib}[1]{%
  \nopagebreak{\raggedleft\footnotesize #1\par}}


% ---------------------------------------------------------------------------- %
% Definicoes comuns ao bibentries em astronomia

\def\aap{A\&A}                % Astronomy and Astrophysics
\def\aapr{A\&A~Rev.}          % Astronomy and Astrophysics Reviews
\def\aaps{A\&AS}              % Astronomy and Astrophysics, Supplement
\def\aj{AJ}                   % Astronomical Journal
\def\ajph{Australian J.~Phys.}% Australian Journal of Physics
\def\alet{Astro.~Lett.}       % Astronomical Letters
\def\ao{Applied Optics}       % Applied Optics
\def\apj{ApJ}                 % Astrophysical Journal
\def\apjl{ApJ}                % Astrophysical Journal, Letters
\def\apjs{ApJS}               % Astrophysical Journal, Supplement
\def\apss{Ap\&SS}             % Astrophysics and Space Science
\def\araa{ARA\&A}             % Annual Review of Astronomy and Astrophysics
\def\pasj{PASJ}
\def\pasp{PASP}
\def\mnras{MNRAS}%
\def\nat{Nature}%
\def\bain{Bull.~Astron.~Inst.~Netherlands}%
\def\baas{BAAS}%
\def\qjras{QJRAS}%            % ROYAL ASTRON. SOC. QUART. JRN}
\def\jqsrt{JQSRT}%            % Journal of Quantitative Spectroscopy & Radiative Transfer}

% ---------------------------------------------------------------------------- %
% Corpo do texto
\begin{document}

\frontmatter 
% cabeçalho para as páginas das seções anteriores ao capítulo 1 (frontmatter)
\fancyhead[RO]{{\footnotesize\rightmark}\hspace{2em}\thepage}
\setcounter{tocdepth}{2}
\fancyhead[LE]{\thepage\hspace{2em}\footnotesize{\leftmark}}
\fancyhead[RE,LO]{}
\fancyhead[RO]{{\footnotesize\rightmark}\hspace{2em}\thepage}

\onehalfspacing  % espaçamento

% ---------------------------------------------------------------------------- %
% CAPA
\thispagestyle{empty}
\begin{center}
    \vspace*{2.3cm}
    Universidade de São Paulo\\
    Instituto de Matemática e Estatística\\
    Bacharelado em Ciência da Computação


    \vspace*{3cm}
    \Large{Isabela Blücher}
    

    \vspace{3cm}
    \textbf{\Large{Filtragem de ruído telúrico \\
    em sinais astronômicos}}
    
       
    \vskip 5cm
    \normalsize{São Paulo}

    \normalsize{2019}
\end{center}

% ---------------------------------------------------------------------------- %
% Página de rosto
%
\newpage
\thispagestyle{empty}
    \begin{center}
        \vspace*{2.3 cm}
        \textbf{\Large{Filtragem de ruído telúrico em sinais astronômicos}}
        \vspace*{2 cm}
    \end{center}

    \vskip 2cm

    \begin{flushright}
	Monografia final da disciplina \\
        MAC0499 -- Trabalho de Formatura Supervisionado.
    \end{flushright}

    \vskip 5cm

    \begin{center}
    Supervisora: Profa. Dra. Paula Rodrigues Teixeira Coelho (IAG-USP)\\
    Co-supervisor: Prof. Dr. Marcelo Gomes de Queiroz (IME-USP)

    \vskip 5cm
    \normalsize{São Paulo}

    \normalsize{2019}
    \end{center}
\pagebreak


\clearpage

\pagenumbering{roman}     % começamos a numerar 

%% % ---------------------------------------------------------------------------- %
% Agradecimentos:
% Se o candidato não quer fazer agradecimentos, deve simplesmente eliminar esta página 
\chapter*{Agradecimentos}
À Universidade de São Paulo e ao Instituto de Matemática e Estatística pela oportunidade de cursar o Bacharelado em Ciência da Computação e por proporcionar um ambiente questionador e criativo para o desenvolvimento de pesquisas acadêmicas.

Aos meus orientadores Profa. Dra. Paula Rodrigues Teixeira Coelho e Prof. Dr. Marcelo Gomes de Queiroz pela oportunidade de elaborar um trabalho que mescla minhas áreas preferidas do conhecimento e pelo paciente trabalho de ensino e revisão que fizeram comigo ao longo do ano.

À Profa. Dra. Nina Sumiko Tomita Hirata, por coordenar a disciplina MAC0499 e pelo suporte no meu envolvimento com a Astronomia e o IAG durante a graduação.

À minha família e amigos pelo apoio emocional, por me incentivarem a sempre dar o meu melhor e continuar estudando assuntos que me interessam.

A todos que direta ou indiretamente fizeram parte da minha formação, o meu muito obrigada.

% ---------------------------------------------------------------------------- %
% Resumo
\chapter*{Resumo}

Na área da astronomia observacional, um dos principais dados usados em pesquisa são os espectros estelares, devido à grande quantidade de informação sobre a estrela contida nele. A maioria das observações estelares capturadas por espectrógrafos são feitas a partir do solo e estão sujeitas à interferência da atmosfera, denominada contaminação telúrica. Este trabalho usa dados estelares reais e dados telúricos sintéticos para testar o método de correção telúrica atual e usa o algoritmo \textit{Dynamic Time Warping} como uma abordagem para propor uma melhoria na correção dos espectros. Os resultados mostram que o algoritmo não performa idealmente \revise{quando aplicados nos espectros estelar e telúrico inteiros}, mas que poderia potencialmente ser aplicado \revise{em intervalos menores, para} linhas de absorção bem separadas.
\\

\noindent \textbf{Palavras-chave:} espectroscopia astronômica, contaminação telúrica, processamento de sinais digitais, realinhamento de sinais.

% ---------------------------------------------------------------------------- %
% Abstract
\chapter*{Abstract}

In the context of observational astronomy, one of the main data used in research are stellar spectra, due to the amount of information it contains about the observed star. Most stellar observations are captured by ground-based spectrographs and \revise{therefore} are subject to atmospheric interference, \revise{commonly}  called telluric contamination. This work uses real stellar data and synthetic telluric data to test the currently used method for telluric correction and uses the Dynamic Time Wapring algorithm as an approach to potentially improve the current spectral correction. The results show that the algorithm does not perform optimally when applied between the full stellar and telluric spectra, but could potentially be used \revise{on smaller regions,} on well separated absorption lines.  
\\

\noindent \textbf{Keywords:} astronomical spectroscopy, telluric contamination, digital signal processing, signal realignment.


% ---------------------------------------------------------------------------- %
% Sumário
\tableofcontents    % imprime o sumário




%% % ---------------------------------------------------------------------------- %
%% \chapter{Lista de Abreviaturas}
%% \begin{tabular}{ll}
%%          CFT         & Transformada contínua de Fourier (\emph{Continuous Fourier Transform})\\
%%          DFT         & Transformada discreta de Fourier (\emph{Discrete Fourier Transform})\\
%%         EIIP         & Potencial de interação elétron-íon (\emph{Electron-Ion Interaction Potentials})\\
%%         STFT         & Tranformada de Fourier de tempo reduzido (\emph{Short-Time Fourier Transform})\\
%% \end{tabular}

%% % ---------------------------------------------------------------------------- %
%% \chapter{Lista de Símbolos}
%% \begin{tabular}{ll}
%%         $\omega$    & Frequência angular\\
%%         $\psi$      & Função de análise \emph{wavelet}\\
%%         $\Psi$      & Transformada de Fourier de $\psi$\\
%% \end{tabular}

%% % ---------------------------------------------------------------------------- %
%% % Listas de figuras e tabelas criadas automaticamente
%% \listoffigures            
%% \listoftables            



% ---------------------------------------------------------------------------- %
% Capítulos do trabalho
\mainmatter

% cabeçalho para as páginas de todos os capítulos
\fancyhead[RE,LO]{\thesection}

% \singlespacing              % espaçamento simples
\onehalfspacing            % espaçamento um e meio


\input 01-cap-introducao        % associado ao arquivo: 'cap-introducao.tex'
\input 02-cap-fundamentacao-teorica

\input 03-cap-desenvolvimentos  % associado ao arquivo: 'cap-desenvolvimento.tex'
% \input 04-cap-implementacao
% \input cap-resultados
\input 04-cap-conclusoes        % associado ao arquivo: 'cap-conclusoes.tex'
% \input 05-cap-perspectivas        % associado ao arquivo: 'cap-conclusoes.tex'

% cabeçalho para os apêndices
\renewcommand{\chaptermark}[1]{\markboth{\MakeUppercase{\appendixname\ \thechapter}} {\MakeUppercase{#1}} }
\fancyhead[RE,LO]{}
\appendix

% \include{ape-exemplo}      % associado ao arquivo: 'ape-conjuntos.tex'


% ---------------------------------------------------------------------------- %
% Bibliografia
\backmatter \singlespacing   % espaçamento simples
\bibliographystyle{plainnat-ime} % citação bibliográfica textual
\bibliography{bibliografia}  % associado ao arquivo: 'bibliografia.bib'


%%%  ---------------------------------------------------------------------------- %
%% % Índice remissivo
%% \index{TBP|see{periodicidade região codificante}}
%% \index{DSP|see{processamento digital de sinais}}
%% \index{STFT|see{transformada de Fourier de tempo reduzido}}
%% \index{DFT|see{transformada discreta de Fourier}}
%% \index{Fourier!transformada|see{transformada de Fourier}}

%% \printindex   % imprime o índice remissivo no documento 

\end{document}
